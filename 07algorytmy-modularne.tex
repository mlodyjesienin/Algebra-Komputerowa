\documentclass{beamer}

\usepackage{polski}
\usepackage{pgf,tikz, tikz-cd}
\usepackage{tgheros}       
\usepackage{bm}

\usetheme{BIT}

\usepackage[utf8]{inputenc}
\usepackage{amssymb,amsmath,amsthm}
\usepackage[backend=bibtex,style=numeric]{biblatex}
\addbibresource{literature.bib}

\usetikzlibrary{arrows}
\usetikzlibrary{shapes,decorations}

\newcommand{\zero}{\mathbf{0}}
\newcommand{\one}{\mathbf{1}}
\newcommand{\ord}{\textrm{ord}}
\newcommand{\II}{\mathnormal{I}}
\newcommand{\JJ}{\mathnormal{J}}
\newcommand{\NWD}{\rm{NWD}}
\newcommand{\NWW}{\rm{NWW}}

\let\phi\varphi
\renewcommand{\epsilon}{\bm{\varepsilon}}

\title{Algebra Komputerowa}
\subtitle{Algorytmy Modularne \cite{ComputerAlgebra,LCM}}
\author{Filip Zieli\'nski}
\date{2025}
 
\begin{document}
\begin{frame}
    \titlepage
\end{frame}
 
\begin{frame}{Spis Treści}
    \tableofcontents
\end{frame}

\section{Chińskie Twierdzenie o Resztach}
\begin{frame}{Układ Kongruencji}
    \begin{block}{Problem}
        Rozważmy układ kongruencji 
        $$
        \begin{cases}
            a \equiv 1 \pmod{5} \\
            a \equiv 6 \pmod{7} \\
            a \equiv 5 \pmod{9}
        \end{cases}
        $$
    \end{block}
    pojawiają się naturalne pytania 
    \begin{itemize}
        \item Czy układ ma rozwiązanie i czy da się je znaleźć?
        \item Jeśli tak, to ile jest tych rozwiązań?
        \item Jak znaleźć wszystkie rozwiązania?
    \end{itemize}
\end{frame}

\begin{frame}{CRT - Wersja Podstawowa}
    \begin{block}{Chińskie Twierdzenie o Resztach (CRT)}
        Niech $R$ będzie pierścieniem Euklidesowym oraz $m_1, \ldots ,m_k, a_1, \ldots a_k \in R$. Oznaczmy $M = \prod_{i=1}^{k} m_i$.
        Jeżeli $m_1, \ldots, m_k$ są parami względnie pierwsze to układ 
        $$
        \begin{cases}
            a \equiv a_1 \pmod{m_1} \\ 
            a \equiv a_2 \pmod{m_2} \\
            \vdots \\ 
            a \equiv a_k \pmod{m_k}
        \end{cases} 
        $$
        ma dokładnie jedno rozwiązanie $ \in R/ \langle M \rangle $ (dokładnie jedno rozwiązenie modulo $M$.)
    \end{block}
\end{frame}

\begin{frame}{CRT - Wersja Zaawansowana}
    \begin{block}{Chińskie Twierdzenie o Resztach (CRT)}
        Niech $R$ będzie pierścieniem oraz $\II_1, \ldots \II_k$ parami względnie pierwszymi ideałami w tym pierścienu.
        Wtedy 
        $$ R/(\II_1 \cap \ldots \cap \II_k) \cong R/\II_1 \times \ldots \times R/ \II_k. $$
    \end{block}
\end{frame}

\section{Układy Kongruencji}
\begin{frame}{Zawężenie do Pierścieni Euklidesowych}
    \begin{alertblock}{Uwaga}
        Oba prezentowane algorytmy, bazują na możliwości wyznaczenia współczynników Bezouta z Rozszerzonego
        Algorytmu Euklidesa, zatem wprost odnoszą się tylko do pierścieni Euklidesowych (standardowo dla nas $\mathbb{Z}, k[x]$)
    \end{alertblock}
\end{frame}

\begin{frame}{Algorytm Lagrange'a}
    \textbf{Założenia}: pierścień euklidesowy $R$.

    \textbf{Wejście:}
    \begin{itemize}
        \item $a_1, \ldots , a_k \in R$
        \item $m_1, \ldots , m_k \in R$
    \end{itemize}
    \textbf{Wyjście:}
    \begin{itemize}
        \item $a : a \equiv a_i \pmod{m_i}$ dla każdego $i=1,\ldots,k$.
    \end{itemize}
    \textbf{Kroki:}
    \begin{enumerate}
        \item Dla każdego $i \in \{1, \ldots k\}, j \in \{i+1,\ldots,k\} $ wyznacz $\alpha_{ij},\alpha_{ji}$ takie, że 
        $\alpha_{ij}m_i + \alpha_{ji}m_j = 1$ (współczynniki Bezouta).
        \item Zdefiniuj $M := \prod_{i=1}^{k}m_i.$ 
        \item Dla każdego $i \in \{1,\ldots,k\}$ wyznacz $A_i := \dfrac{M}{m}\prod_{j=1, j\neq i}^{k}\alpha_{ji}.$
        \item Zwróć $a := \sum_{i=1}^{k}a_i A_i \pmod{M}.$ 
    \end{enumerate}
\end{frame}

\begin{frame}{Wady Algorytmu Lagrange'a}
    \begin{alertblock}{Uwaga}
        Współczynniki $A_i$ z poprzedniego algorytmu bardzo szybko rosną oraz nie jest to algorytm "przyrostowy"
    \end{alertblock}
\end{frame}

\begin{frame}{Układy dwóch kongruencji}
    \textbf{Wejście:}:
    \begin{itemize}
        \item $a_1, a_2 \in R$
        \item $m_1, m_2 \in R$
    \end{itemize}
    \textbf{Wyjście:}
    \begin{itemize}
        \item $a$ taki, że $a \equiv a_1 \pmod{m_1} \land a \equiv a_2 \pmod{m_2}$
    \end{itemize}
    \textbf{Kroki:}
    \begin{enumerate}
        \item Wyznacz $\alpha, \beta$ takie, że $\alpha m_1 + \beta m_2 = 1.$
        \item Zdefiniuj $r := mod(a_1,m_1)$
        \item Wyznacz $b := (a_2 -r)\alpha$
        \item Wyznacz $s := b \pmod{m_2}$ 
        \item Zwróć $r + sm_1$ 
    \end{enumerate}
\end{frame}

\begin{frame}{Algorytm przyrostowy Newtona}
    \textbf{Wejście:}
    \begin{itemize}
        \item $a_1, \ldots , a_k \in R$
        \item $m_1, \ldots , m_k \in R$
    \end{itemize}
    \textbf{Wyjście:}
    \begin{itemize}
        \item $a : a \equiv a_i \pmod{m_i}$ dla każdego $i=1,\ldots,k$.
    \end{itemize}
    \textbf{Kroki:}
    \begin{enumerate}
        \item Zdefiniuj $M:=1, A:=1$
        \item dla każdego $i \in \{1,\ldots, k\}$ \begin{itemize}
            \item Używając poprzedniego algorytmu, znajdź b takie, że $b \equiv A \pmod{M} \land b \equiv a_i \pmod{m_i}$.
            \item Zamień $A := b$ oraz $M:=M \cdot m[i].$
        \end{itemize}
        \item Zwróć $A$.
    \end{enumerate}
\end{frame}

\section{Potęgowanie Modulo}
\begin{frame}{Szybkie potęgowanie modulo}
    \textbf{Wejście:}
    \begin{itemize}
        \item $a \in \mathbb{Z}, n,m \in \mathbb{N}$ 
    \end{itemize}
    \textbf{Wyjście:}
    \begin{itemize}
        \item $a^{n} \pmod{m}$
    \end{itemize}
    \textbf{Kroki:}
    \begin{enumerate}
        \item Zdefiniuj $(d_k \ldots d_0)_2$ jako binarny zapis wykładnika $n$. 
        \item Zainicjalizuj $r:=1$, $b:= a \pmod{m}$ 
        \item Dla każdego $i \in \{1, \ldots, k\}$
        \item \begin{itemize}
            \item Jeżeli $d_i = 1$ to podstaw $r :=  r\cdot b \pmod{m}$
            \item Podstaw $b := b^2 \pmod{m}$ 
        \end{itemize}
        \item Zwróć $r$
    \end{enumerate}
\end{frame}

\begin{frame}{Małe Twierdzenie Fermata}
    \begin{block}{Twierdzenie (Fermat, 1640)}
        Jeżeli $p$ jest liczbą pierwszą, to dla każdego $a \in \mathbb{Z}$ zachodzi 
        $$a^{p-1} \equiv \begin{cases}
            0 \pmod{p}, \quad p \mid a \\ 
            1 \pmod{p}, \quad p \nmid a 
        \end{cases}$$
    \end{block}
    \pause 
    \begin{alertblock}{Obserwacja}
        To twierdzenie, pozwala często przyspieszyć potęgowanie modularne (zmniejszyć błyskawicznie wykładnik). Niestety tylko i wyłącznie, gdy potęgujemy modulo liczba pierwsza.
    \end{alertblock}
\end{frame}

\begin{frame}{Funkcja $\phi$ Eulera}
    \begin{block}{Definicja}
        Funkcją Eulera nazywamy funkcję $\phi : \mathbb{N} \rightarrow \mathbb{N}$ zdefiniowaną jako
        $$\phi(n) = \# \{a \in \{0,\ldots, n-1\} \mid NWD(a,n) = 1\}$$
        Innymi słowami $\phi(n)$ to liczba jedności w pierścieniu ilorazowym $\mathbb{Z}/\langle n \rangle.$
    \end{block}
    \pause 
    \begin{alertblock}{Obserwacja}
        Jeżeli $p$ jest liczba pierwszą, to $\phi(p) = p-1$
    \end{alertblock}
    \begin{block}{Twierdzenie}
        \begin{itemize}
            \item Jeżeli $p$ jest liczbą pierwszą, to $\phi(p^k) = p^k - p^{k-1}$
            \item Jeżeli $\NWD(m,n) = 1$ to $\phi(mn) = \phi(m) \cdot \phi(n)$
        \end{itemize}
    \end{block}
\end{frame}

\begin{frame}{Twierdzenie Eulera}
    \begin{block}{Twierdzenie (Euler, 1736)}
        Niech $m >1$ będzie liczbą całkowitą dodatnią. Jeżeli $a \in \mathbb{Z}$ jest względnie pierwsze z $m$ to zachodzi 
        $$a^{\phi(m)} \equiv 1 \pmod{m}$$
    \end{block}
    \pause 
    \begin{alertblock}{Obserwacja}
        To twierdzenie również bardzo przyspiesza potęgowanie modularne.
    \end{alertblock}
\end{frame}

\section{Wyznacznik Macierzy Całkowitoliczbowej}
\begin{frame}{Problemy Eliminacji Gaussa}
    \begin{alertblock}{Obserwacja}
        Rozważmy macierz kwadratową $M \in \mathcal{M}_{n}(\mathbb{Z})$.
        W oczywisty sposób $\det M \in \mathbb{Z}$. 
        
        Mimo tego, wykorzystując klasyczne algorytmy liczenia wyznacznika (eliminacja Gaussa) na komputerze 
        z powodu dzielenia możemy natrafić na liczby zmiennoprzecinkowe i końcowy wynik może nie być dokładny. 

        Pomysł: W ciałach skończonych $\mathbb{Z}_p$ operację są dokładne - można policzyć wyznacznik modulo wiele liczb pierwszych, a później z Chińskiego Twierdzenia o Resztach złożyć wiele wyników w jeden ostateczny.
        Dodatkowa zaleta: Operacje modulo są stosunkowo szybkie!
    \end{alertblock}
\end{frame}

\begin{frame}{Wyznacznik Macierzy Modulo}
    \begin{block}{Twierdzenie}
        Niech $M = (m_{ij})\in \mathcal{M}_n(\mathbb{Z})$ będzie macierza kwadratową $n \times n$, a $p$ dowolną liczbą pierwszą.
        Niech $\overline{m}_{ij} = (m_{ij} \pmod{p})$ oraz $\overline{M} = (\overline{m}_{ij})$. Zachodzi 
        $$\det \overline{M} \equiv \det M \pmod{p}$$
    \end{block}
\end{frame}

\begin{frame}{Nierówność Hadamarda}
    \begin{block}{Twierdzenie (Hadamard, 1893)}
        Niech $M = (m_{ij}) \in \mathcal{M}_n(\mathbb{R})$ będzie rzeczywistą macierzą kwadratową. Zachodzi 
        $$ |\det M | \leq \prod_{j=1}^{n}\sqrt{\sum_{i=1}^{n}m_{ij}^{2}}$$
    \end{block}
    \pause 
    \begin{alertblock}{Wniosek}
        Jeżeli rozważymy $B > 0$ takie, że $ | m_{ij} | \leq B$ dla wszystkich $i,j \leq n$, to zachodzi 
        $$ | \det M| \leq B^n \sqrt{n^n}$$ 
    \end{alertblock}
\end{frame}

\begin{frame}{Algorytm Modularny Gaussa}
    \textbf{Wejście:}
    \begin{itemize}
        \item $M = (m_{ij})$ macierz  całkowitoliczbowa wymiaru $n \times n$.
    \end{itemize}
    \textbf{Wyjście:}
    \begin{itemize}
        \item $\det M$.
    \end{itemize}
    \textbf{Kroki:}
    \begin{enumerate}
        \item Wyznacz $B \geq |m_{ij}||$ dla każdego $i,j \leq n$. 
        \item Znajdź liczby pierwsze $p_1, \ldots p_k \geq 3$ takie, że $m := p_1 \cdots p_k > 2B^n\sqrt{n^n}$.
        \item Dla każdego $i \in \{1, \ldots, k\}$ oblicz $d_i := det(M \pmod{p_i}) \in \mathbb{Z}_p$.
        \item Rozwiąż układ kongruencji $d \equiv d_i \pmod{p_i}$ dla $i \in \{1,\ldots,k\}$.
        \item Jeśli $d > m/2$ to podstaw $d := d-m$.
        \item Zwróć $d$. 
    \end{enumerate}
\end{frame}
\begin{frame}{Źródła}
    Prezentacja jest mocno oparta o wykład autorstwa \textit{Przemysława Koprowskiego}, który można obejrzeć pod tym 
    \href{https://www.youtube.com/watch?v=O4e0N-0glUw}{linkiem}
    \printbibliography
\end{frame}

\begin{frame}
    \centering 
    \LARGE Pytania, wątpliwości, uwagi ? 
\end{frame}

\end{document}