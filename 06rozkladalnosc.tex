\documentclass{beamer}

\usepackage{polski}
\usepackage{pgf,tikz, tikz-cd}
\usepackage{tgheros}       
\usepackage{bm}

\usetheme{BIT}

\usepackage[utf8]{inputenc}
\usepackage{amssymb,amsmath,amsthm}
\usepackage[backend=bibtex,style=numeric]{biblatex}
\addbibresource{literature.bib}

\usetikzlibrary{arrows}
\usetikzlibrary{shapes,decorations}

\newcommand{\zero}{\mathbf{0}}
\newcommand{\one}{\mathbf{1}}
\newcommand{\ord}{\textrm{ord}}
\newcommand{\II}{\mathnormal{I}}
\newcommand{\JJ}{\mathnormal{J}}
\newcommand{\NWD}{\rm{NWD}}
\newcommand{\NWW}{\rm{NWW}}

\let\phi\varphi
\renewcommand{\epsilon}{\bm{\varepsilon}}

\title{Algebra Komputerowa}
\subtitle{Pierścienie z jednoznacznością rozkładu \cite{Computative1,Gleichgewicht}}
\author{Filip Zieli\'nski}
\date{2025}
 
\begin{document}
\begin{frame}
    \titlepage
\end{frame}
 
\begin{frame}{Spis Treści}
    \tableofcontents
\end{frame}

\section{Podzielność}
\begin{frame}{Założenia}
    \begin{alertblock}{Uwaga}
        Przypominam, że mówiąc pierścień mamy na myśli pierścień przemienny z jedynką. \\
    \end{alertblock}
    \pause
    \begin{alertblock}{Uwaga}
        Dodatkowo, w tym rozdziale zawężamy nasze rozważania do pierścieni całkowitych. Mówiąc pierścień, mamy na myśli pierścień całkowity przemienny z jedynką.
    \end{alertblock}
\end{frame}

\begin{frame}{Relacja podzielności}
    Niech $R$ będzie pierścieniem oraz $a,b$ elementami tego pierścienia.
    \begin{block}{Definicja}
        Mówimy, że element $a$ \textit{dzieli} $b$, jeśli istnieje taki element $c \in R$, że $ac = b$.
        Tak więc 
        $$ a \mid b \Leftrightarrow \exists c \in R : b = ac$$
        Gdy $a \mid b$ mówimy, że "$a$ jest dzielnikiem elementu $b$" lub "$b$ jest podzielne przez $a$" lub "$b$ jest wielokrotnością $a$". Jeśli $a$ nie dzieli $b$ to piszemy $a \nmid b$.
    \end{block}
\end{frame}

\begin{frame}{Własności podzielności}
    \begin{alertblock}{Obserwacja}
        Relacja $\mid$ określona na pierścieniu $R$ jest relacją zwrotnią i przechodnią.
    \end{alertblock}
    \begin{block}{Twierdzenie}
        Dla dowolnych $a,b,c,d \in R$ zachodzi 
        \begin{enumerate}
            \item $\one \mid a$ 
            \item $a \mid b \Rightarrow a \mid bc$ 
            \item $a \mid b \land a \mid c \Rightarrow a \mid b \pm c$
            \item $a \mid b \Rightarrow ac \mid bc$
            \item $a \mid b \land c \mid d \Rightarrow ac \mid bd$
        \end{enumerate}
    \end{block}
\end{frame}

\begin{frame}{Relacja stowarzyszenia}
    \begin{block}{Definicja}
        Mówimy, że elementy $a,b \in R$ są stowarzyszone, co zapisujemy $a \sim b$, jeśli $a \mid b \land b \mid a$. 
    \end{block}
    \begin{exampleblock}{Obserwacja}
        Relacja stowarzyszenia, jest relacją równoważności.
    \end{exampleblock}
    \pause 
    \begin{block}{Definicja}
        Element $u$ pierścienia $R$ nazywamy jednością, jeżeli istnieje w pierścieniu $R$ do niego element odwrotny.
    \end{block}
    \begin{block}{Definicja}
        Zbiór wszystkich jedności w pierścieniu $R$ nazywamy \textit{zbiorem elementów odwracalnych} i oznaczamy przez $U_R$.
    \end{block}
\end{frame}

\begin{frame}{Zbiór jedności}
    \begin{alertblock}{Obserwacja}
        Zbiór elementów odwracalnych pierścienia $R$ jest grupą względem ich mnożenia.
    \end{alertblock}
    \pause 
    \begin{alertblock}{Obserwacja}
        Zachodzi 
        $U_R = \{ a\in R : a \sim \one\}$,
        czyli jednoścu to dokładnie elementy stowarzyszone z jedynką. 
    \end{alertblock}
    \pause 
    \begin{block}{Twierdzenie}
        Jeśli $a,b$ są elementami pierścienia $R$, to 
        $$a \sim b \Leftrightarrow \exists \epsilon \in U_R : b = a \epsilon.$$
    \end{block}
\end{frame}

\section{Pierścienie z jednoznacznościa rozkładu}
\begin{frame}{Rozkład elementu}
    \begin{block}{Definicja}
        Niech $a \in R \setminus \{ \zero \}$. Rozkładem elementu $a$ na czynniki nazywamy każde przedstawienie go w postaci iloczynu $a = a_1 a_2 \ldots a_n$.
    \end{block}
    \pause 
    \begin{block}{Definicja}
        Różny od zera element $a$ pierścienia $R$ nazywamy \alert{rozkładalnym}, jeśli da  się go przestawić jako iloczyn $a_1, a_2$ gdzie $a_1$ oraz $a_2$ są elementami nieodwracalnymi. \\ 
        Jeżeli element nie jest zerem, nie jest jednością oraz nie jest rozkładalny to nazywamy go \alert{nierozkładalnym.}
    \end{block}
\end{frame}

\begin{frame}{Jednoznaczność rozkładu}
    \begin{block}{Definicja}
        Niech $R$ będzie pierścieniem. 
        \begin{itemize}
            \item Pierścień $R$ nazywamy pierścieniem z rozkładem, gdy każdy niezerowy i nieodwracalny element tego pierścienia można przedstawić w postaci iloczynu elementów nierozkładalnych.
            \item Pierścień $R$ nazywmy \alert{pierścieniem z jednoznacznościa rozkładu} (lub 
            \textit{pierścieniem gaussowskim}, lub \textit{UFD}), gdy każdy niezerowy i nieodwracalny element tego pierścienia można
            przedstawić w postaci iloczynu elementów nierozkładalnych w sposób jednoznaczny z dokładnością do stowarzyszenia.
        \end{itemize}
    \end{block}
\end{frame}

\begin{frame}{Elementy pierwsze}
    \begin{block}{Definicja}
        Niech $p$ będzie elementem pierścienia $R$. element $p$ nazywamy \alert{pierwszym}, jeżeli zachodzi 
        $$p \mid ab \Rightarrow p \mid a \lor p \mid b.$$
    \end{block}
    \pause 
    \begin{block}{Twierdzenie}
        Jeżeli $p$ jest elementem pierwszym pierścienia $R$ to $p$ jest elementem nierozkładalnym. 
    \end{block}
    \pause 
    \begin{alertblock}{Uwaga}
        Twierdzenie odwrotne nie jest prawdziwe. Element nierozkładalny nie musi być pierwszy.
    \end{alertblock}
\end{frame}

\begin{frame}{Elementy pierwsze, a nierozkładalne}
    \begin{block}{Twierdzenie}
        Niech $R$ będzie pierścieniem z rozkładem. Następujące warunki są równoważne
        \begin{itemize}
            \item $R$ jest pierścieniem z jednoznacznościa rozkładu, 
            \item Każdy element nierozkładalny w $R$ jest pierwszy.
        \end{itemize}
    \end{block}
\end{frame}

\begin{frame}
    \begin{block}{Twierdzenie}
        Niech $R$ będzie pierścieniem z jednoznacznością rozkładu. Wtedy $R[x]$ jest pierścieniem z jednoznacznością rozkładu.
    \end{block}
    \pause 
    \begin{alertblock}{Wniosek}
        Niech $R$ będzie pierścieniem z jednoznacznościa rozkładu. Wtedy $R[x_1, \ldots x_n]$ jest pierścieniem z jednoznacznościa rozkładu. 
    \end{alertblock}
\end{frame}

\section{NWD i NWW}
\begin{frame}{Rozkład kanoniczny}
    Niech $R$ będzie pierścieniem z jednoznacznościa rozkładu.
    \begin{alertblock}{Obserwacja}
        Niech $\mathcal{P} \subseteq R$ będzie zbiorem reprezentantów klas abstrakcji względem relacji stowarzyszenia
        wyznaczonych przez elementy nierozkładalne to każdy element $a \in R$ ma jednoznaczne przedstawienie postaci 
        $$a = \epsilon \prod_{p \in \mathcal{P}} p^{\alpha_{p}}$$
        gdzie $\epsilon \in U_R$ oraz $alpha_p \in \mathbb{N} \cup \{0\}$, z tym, że $\alpha_p = 0$ dla prawie wszystkich $p$. Takie przedstawienie nazywamy \alert{rozkładem kanonicznym}.
    \end{alertblock}
\end{frame}

\begin{frame}{NWD i NWW}
    Niech $R$ będzie pierścieniem z jednoznacznościa rozkładu oraz $a,b \in R$ będą elementami pierścienia z rozkładami kanonicznymi odpowiednio
    $\epsilon_1 \prod_{p \in \mathcal{P}} p^{\alpha_{p}}$ oraz $\epsilon_2 \prod_{p \in \mathcal{P}} p^{\beta_{p}}$. 
    \begin{alertblock}{Twierdzenie}
        $b \mid a$ wtedy i tylko wtedy, gdy $\beta_p \leq \alpha_p$ dla każdego $p \in \mathcal{P}$.
    \end{alertblock}
    \pause
    \begin{block}{Definicja}
        \begin{itemize}
            \item\alert{Największym wspólnym dzielnikiem} $a,b$ nazywamy $\NWD (a,b) = \prod_{p \in \mathcal{P}} p^{min\{\alpha_{p}, \beta_{p}\}}$
            \item \alert{Najmniejszą wspólną wielokrotnością} $a,b$ nazywamy $\NWW (a,b) = \prod_{p \in \mathcal{P}} p^{max\{\alpha_{p}, \beta_{p}\}} $
        \end{itemize}
    \end{block}
\end{frame}

\begin{frame}{Rozszerzenie definicji}
    \begin{block}{Definicja}
        Niech $R$ będzie pierścieniem z jednoznacznościa rozkładu i $a_1, \ldots a_m$ elementami tego pierścienia.
        Dla $m >2$ pojęcie największego wspólnego dzielnika i najmniejszej wspólnej wielokrotności definiujemy rekurencyjnie jako
        \begin{align*}
            \NWD(a_1 , \ldots , a_m) & = \NWD(\NWD(a_1, \ldots , a_{m-1}), a_m) \\
            \NWW(a_1 , \ldots , a_m) & = \NWW(\NWW(a_1, \ldots , a_{m-1}), a_m)
        \end{align*}
    \end{block}
\end{frame}

\begin{frame}{Własności NWD i NWW}
    \begin{alertblock}{Obserwacja}
        Dla elementów $a_1, \ldots, a_m$, gdzie $a_i$ ma rozkład kanoniczny $\epsilon_i \prod_{p \in \mathcal{P}} p^{\alpha_{p_{i}}}$ zachodzi 
        \begin{align*}
            \NWD(a_1 , \ldots , a_m) & = \prod_{p \in \mathcal{P}} p^{min\{\alpha_{p_1}, \ldots , \alpha_{p_{m}}\}}\\
            \NWW(a_1 , \ldots , a_m) & = \prod_{p \in \mathcal{P}} p^{max\{\alpha_{p_1}, \ldots , \alpha_{p_{m}}\}}.
        \end{align*}
    \end{alertblock}
\end{frame}

\begin{frame}{Własności NWD i NWW}
    \begin{alertblock}{Obserwacja}
        Tak zdefiniowane $\NWD$ oraz $\NWW$ zależy od wyboru reprezentantów klas abstrakcji elementów nierozkładalnych.
        Rozważmy dwa różne zbiory tych reprezentantów  $\mathcal{P}_1 , \mathcal{P}_2 \subseteq R$ oraz elementy pierścienia $a_1, \ldots, a_m$. 
        Zauważmy, że 
        \begin{align*}
            \NWD_{\mathcal{P}_1} (a_1 , \ldots a_m) & \sim  \NWD_{\mathcal{P}_2} (a_1 , \ldots a_m) \\ 
            \NWW_{\mathcal{P}_1} (a_1 , \ldots a_m) & \sim \NWW_{\mathcal{P}_2} (a_1 , \ldots a_m).
        \end{align*}
        Z tego powodu można myśleć, że dla danych elementów $\NWD$ jest dokładnie jedno oraz $\NWW$ jest dokładnie jedno.
    \end{alertblock}
\end{frame}

\begin{frame}{Charakterystyka NWD i NWW}
    \begin{block}{Twierdzenie}
        Niech $R$ będzie pierścieniem z jednoznacznością rozkładu oraz niech $a_1, \ldots a_m$ będą elementami tego pierścienia. $d = \NWD(a_1, \ldots, a_m)$ wtedy i tylko wtedy, gdy
        \begin{itemize}
            \item $\forall i=1,\ldots, m \quad d \mid a_i$, 
            \item $\forall c \in R  \quad [(\forall i=1 , \ldots , m \quad c \mid a_i)  \Rightarrow c \mid d]$.
        \end{itemize}
    \end{block}
    \pause 
    \begin{block}{Twierdzenie}
        Niech $R$ będzie pierścieniem z jednoznacznością rozkładu oraz niech $a_1, \ldots a_m$ będą elementami tego pierścienia. $d = \NWW(a_1, \ldots, a_m)$ wtedy i tylko wtedy, gdy
        \begin{itemize}
            \item $\forall i=1,\ldots, m \quad a_i \mid d$, 
            \item $\forall c \in R  \quad [(\forall i=1 , \ldots , m \quad a_i \mid c)  \Rightarrow d \mid c]$.
        \end{itemize}
    \end{block}
\end{frame}

\begin{frame}{Własności NWD i NWW}
    \begin{block}{Definicja}
        Elementy $a,b$ pierścienia z jednoznacznościa rozkładu $R$, nazywamy \alert{względnie pierwszymi}, jeżeli $\NWD(a,b) = \one$
    \end{block}
    \begin{block}{Twierdzenie}
        Niech $R$ będzie pierścieniem z jednoznacznością rozkładu oraz $a_1 , \ldots, a_m \in R$ elementami tego pierścienia. Zachodzi
        \begin{itemize}
            \item $\langle \NWW(a_1, \ldots , a_m) \rangle =  \langle a_1 \rangle \cap \cdots \cap \langle a_m \rangle$
            \item $\NWD(a_1, a_2) = \dfrac{a_1 a_2}{\NWW(a_1,a_2)}$
        \end{itemize}
    \end{block}
\end{frame}

\section{Pierścienie ideałów głównych}
\begin{frame}{Ideał główny}
    Niech $R$ będzie pierścieniem.
    \begin{block}{Definicja}
        Ideał $\II$ pierścienia $R$ nazywamy \alert{głównym}, jeśli istnieje taki element $a \in R$, że $\II = \langle a \rangle.$
    \end{block}
    \pause 
    \begin{block}{Definicja}
        $R$ nazywamy \alert{pierścieniem ideałów głównych} (lub \textit{pierścieniem głównym}, lub \textit{PID}), jeżeli każdy ideał tego pierścienia jest główny. 
    \end{block}
\end{frame}

\begin{frame}{PID, a UFD}
    \begin{block}{Twierdzenie}
        Niech $R$ będzie pierścieniem ideałów głównych. Wtedy $R$ jest pierścieniem z jednoznacznościa rozkładu.        
    \end{block}
    \pause 
    \begin{block}{Twierdzenie}
        Niech $R$ będzie pierścieniem ideałów głównych oraz $a_1, \ldots , a_m$ elementami tego pierścienia oraz $d = \NWD(a_1, \ldots , a_m)$. Zachodzi 
        $$\langle a_1 , \ldots , a_m \rangle = \langle d \rangle $$
    \end{block}
    \pause 
    \begin{alertblock}{Wniosek}
        Niech $a_1, \ldots a_m$ będą elementami pierścienia ideałów głównych $R$ oraz $d$ największym wspólnym dzielnikiem tych elementów. Istnieją takie $r_1, \ldots r_m \in R$, że 
        $r_1 a_1 + \ldots + r_m a_m = d$.
    \end{alertblock}
\end{frame}

\section{Pierścienie Euklidesowe}
\begin{frame}{Pierścienie Euklidesowe}
    \begin{block}{Definicja}
        Pierścień $R$ nazywamy \alert{pierścieniem euklidesowym}, jeśli określona jest funkcja $N: R\setminus \{ \zero\} \rightarrow \mathbb{N} \cup \{0\}$ zwaną normą, gdzie zachodzi 
        $$ \forall a,b \in R, b \neq \zero  \ \exists q,r \in R : [(a = bq + r) \land (N(r) < N(b) \lor r = \zero)].$$
        element $r$ w powyższym wzorze nazywamy \textit{resztą}.
    \end{block}
\end{frame}

\begin{frame}{Pierścienie Euklidesowe, a PID}
    \begin{block}{Twierdzenie}
        Niech $R$ będzie pierścieniem euklidesowym. Wtedy $R$ jest też pierścieniem ideałów głównych. 
    \end{block}
    \begin{alertblock}{Uwaga}
        Twierdzenie odwrotne nie zachodzi. Np. $\mathbb{Z}[\frac{1}{2}(1 + \sqrt{-19})]$ jest pierścieniem ideałów głównych, natomiast nie jest pierścieniem euklidesowym. 
        Dowód można zobaczyć pod tym \href{https://webspace.maths.qmul.ac.uk/r.a.wilson/MTH5100/PIDnotED.pdf}{\alert{linkiem}}
    \end{alertblock}
\end{frame}

\begin{frame}{Przykłady}
    \begin{exampleblock}{Przykład}
        Pierścieniami euklidesowymi są np. 
        \begin{itemize}
            \item $\mathbb{Z}$ z norma zadaną przez $N(a) = |a|$,
            \item $K[x]$, gdzie $K$ jest ciałem, a norma jest  zadana przez $N(f) = deg(f)$,
            \item $\mathbb{Z}[\sqrt{-1}]$ z normą zadaną przez $N(a + bi) = a^2 + b^2$.
        \end{itemize}
    \end{exampleblock}
\end{frame}

\begin{frame}{Źródła}
    \begin{itemize}
        \item[] \alert{Paweł Gładki}, Uniwersytet Śląski - Podstawowe pojęcia teorii podzielności. \href{https://www.math.us.edu.pl/~pgladki/teaching/2011-2012/a3z_w07.pdf}{Pierścienie z jednoznacznym rozkładem}
    \end{itemize}
    \printbibliography
\end{frame}

\begin{frame}
    \centering 
    \LARGE Pytania, wątpliwości, uwagi ? 
\end{frame}

\end{document}