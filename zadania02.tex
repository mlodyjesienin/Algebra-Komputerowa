\documentclass{article}
\usepackage{amsmath, amssymb}
\usepackage{polski}
\usepackage{enumitem}

\newcommand{\ord}{\textrm{ord}}
\let\phi\varphi

\title{Lista Zadań 02 --  Elementy Teorii Grup}
\author{Filip Zieliński}
\date{\today}

\begin{document}

\maketitle

 \textbf{Grupy Cykliczne} 
\begin{enumerate}
    \item Czy cykliczna jest grupa $(\mathbb{Z}, \star)$, gdzie działanie $\star$ określone jest wzorem $a \star b = a + b + 5$.
    \item Udowodnić, że dla $n > 1$ grupa $\mathbb{Z}_n \times \mathbb{Z}_n$ nie jest cykliczna.
    \item Udowodnić, że jeśli NWD$(m,n) > 1$ to grupa $\mathbb{Z}_n \times \mathbb{Z}_m$ nie jest cykliczna.
    \item Udowodnić, że dla każdego $a \in G$ zachodzi $\ord (a) = \ord (a^-1)$.
    \item Udowodnić, że jeśli $G$ jest grupą abelową, to zbiór $T(G) = \{ a \mid a \in G \land \ord (a) < \infty \}$ jest podgrupą grupy $G$. (Nazywamy ją \textit{pogdupą torsyjną grupy} $G$.)
    \item Udowodnić, że jeśli $\phi : G \rightarrow H$ jest homomorfizmem grup oraz $\phi (a) = b$ i $\ord (a) < \infty$ to $\ord (b) \mid \ord (a)$. Udowodnić, że jeśli $\phi$ jest izomorfizmem, to zachodzi $\ord (b) = \ord (a)$. 
    \item Udowodnić, że obraz homomorficzny grupy cyklicznej jest grupą cykliczną.
\end{enumerate}
\end{document}
