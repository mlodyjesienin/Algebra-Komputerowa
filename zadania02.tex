\documentclass{article}
\usepackage{amsmath, amssymb}
\usepackage{polski}
\usepackage{enumitem}

\newcommand{\ord}{\textrm{ord}}
\let\phi\varphi

\title{Lista Zadań 02 --  Elementy Teorii Grup}
\author{Filip Zieliński}
\date{\today}

\begin{document}

\maketitle

\subsubsection*{Grupy Cykliczne}
\begin{enumerate}
    \item Czy cykliczna jest grupa $(\mathbb{Z}, \star)$, gdzie działanie $\star$ określone jest wzorem $a \star b = a + b + 5$.
    \item Udowodnić, że dla $n > 1$ grupa $\mathbb{Z}_n \times \mathbb{Z}_n$ nie jest cykliczna.
    \item Udowodnić, że jeśli $\operatorname{NWD}(m,n) > 1$, to grupa $\mathbb{Z}_n \times \mathbb{Z}_m$ nie jest cykliczna.
    \item Udowodnić, że dla każdego $a \in G$ zachodzi $\ord (a) = \ord (a^{-1})$.
    \item Udowodnić, że jeśli $G$ jest grupą abelową, to zbiór  \\ $T(G) = \{ a \in G \mid \ord (a) < \infty \}$ jest podgrupą grupy $G$. (Nazywamy ją \textit{podgrupą torsyjną grupy} $G$.)
    \item Udowodnić, że jeśli $\phi : G \rightarrow H$ jest homomorfizmem grup oraz $\phi (a) = b$ i $\ord (a) < \infty$ to $\ord (b) \mid \ord (a)$. Udowodnić, że jeśli $\phi$ jest izomorfizmem, to zachodzi $\ord (b) = \ord (a)$.
    \item Udowodnić, że obraz homomorficzny grupy cyklicznej jest grupą cykliczną.
\end{enumerate}

\subsubsection*{Warstwy}
\begin{enumerate}[resume]
    \item  Udowodnić, że zbiór odwrotności elementów z warstwy $aH$ to dokładnie warstwa $Ha^{-1}$.
\end{enumerate}

\subsubsection*{Podgrupy Normalne}
\begin{enumerate}[resume]
    \item Niech $G$ bedzie grupą oraz $H < G$. Udowodnij, że następujące warunki są równoważne
        \begin{itemize}
            \item $\forall a \in G \quad aH = Ha$
            \item $\forall a \in G \quad aHa^{-1} = H$
            \item $\forall a \in G \quad aHa^{-1} \subseteq H$
        \end{itemize}
    \item Centrum grupy $G$ oznaczamyy przez $Z(G)$ i definiujemy jako
        $$Z(G) = \{ g\in G \mid \forall a \in G \ ag = ga\}.$$
        Udowodnić, że $Z(G) \trianglelefteq G$ dla każdej grupy $G$.
    \item Niech $G$ będzie grupą oraz $H \trianglelefteq G$ i $F < G$. Zdefiniujmy zbiór $HF$ jako $HF = \{ hf \in G \mid h \in H \land f \in F\}$ oraz zbiór $FH$ analogicznie. Udowodnić, że
        \begin{itemize}
            \item $HF < G$,
            \item $FH < G$,
            \item jeżeli dodatkowo $F \trianglelefteq G$, to $HF \trianglelefteq G$.
        \end{itemize}
    \item Sprawdzić, że jeśli $S$ jest niżej podanym podzbiorem zbioru $R^*$ oraz  \\  $H = \{ A \in GL_n(\mathbb{R}) \mid \text{det}A \in S\}$ to $H \trianglelefteq G$ \\
        a) $\mathbb{R}^+$, \qquad b) $Q^*$, \qquad c) $Q^{+}$, \qquad d) $\{ 2^k \mid k \in \mathbb{Z} \}$.
    \item Niech będzie dany niepusty zbiór $T$ oraz grupa $G$. Udowodnić, że
        \begin{itemize}
            \item jeśli $H_t < G$ dla każdego $t \in T$, to $\bigcap_{t \in T}H_t < G$,
            \item jeśli $H_t \trianglelefteq G$ dla każdego $t \in T$, to $\bigcap_{t \in T}H_t \trianglelefteq G$.
        \end{itemize}
\end{enumerate}

\end{document}
