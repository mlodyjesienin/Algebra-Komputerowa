\documentclass{article}
\usepackage{amsmath, amssymb}
\usepackage{polski}
\usepackage{enumitem}

\title{Lista Zadań 01 -- Podstawowe własności Struktur Algebraicznych}
\author{Filip Zieliński}
\date{\today}

\begin{document}

\maketitle

W zadaniach 1--4 stwierdź, czy podane działanie jest wewnętrzne.
\begin{enumerate}
    \item Rozważmy zbiór $\mathcal{C}(\mathbb{R})$ funkcji ciągłych o dziedzinie i przeciwdziedzinie rzeczywistej. Definiujemy działanie $+ : \mathcal{C}(\mathbb{R}) \times \mathcal{C}(\mathbb{R}) \rightarrow \mathcal{C}(\mathbb{R})$ jako $(f + g)(x) = f(x) + g(x)$ (sumowanie po wartościach). Czy to działanie jest dobrze zdefiniowane (wewnętrzne)?
    \item Rozważmy zbiór $LF(V,\mathcal{K})$ odwzorowań liniowych przestrzeni wektorowej $V$ nad ciałem $\mathbb{K}$ w samą siebie. Definiujemy działanie $+ : LF(V,\mathbb{K}) \times LF(V, \mathbb{K}) \rightarrow LF(V,\mathbb{K})$ jako dodawanie po wartościach. Czy to działanie jest dobrze zdefiniowane (wewnętrzne)?
    \item Rozważmy zbiór $H((G, +))$ endomorfizmów grupy $G$. Definiujemy działanie $+ : H((G, +)) \times H((G, +)) \rightarrow H((G, +))$ jako dodawanie po wartościach. Czy to działanie jest dobrze zdefiniowane (wewnętrzne)?
    \item Rozważmy zbiór $F\uparrow(\mathbb{R})$ funkcji rosnących o dziedzinie i przeciwdziedzinie rzeczywsitej. Definujemy działanie $ + : F\uparrow(\mathbb{R}) \times F\uparrow(\mathbb{R}) \rightarrow F\uparrow(\mathbb{R})$ jako dodawanie po wartościach. Czy takie działanie jest dobrze zdefiniowane (wewnętrzne)?
\end{enumerate}

W zadaniach 5--10 odnosimy się do grupy $(G,\cdot)$.
\begin{enumerate}[resume]
    \item Wykaż, że istnieje dokładnie jeden element neutralny w $G$.
    \item Wykaż, że istnieje dokładnie jeden element symetryczny dla każdego elementu w $G$.
    \item Wykaż, że dla każdego $a \in G$ zachodzi $ (a^{-1})^{-1} = a$.
    \item Wykaż, że dla każdego $a,b \in G$ zachodzi $ (ab)^{-1} = b^{-1}a^{-1}$.
    \item Wykaż, że jeżeli dla każdego $a \in G$ zachodzi $a a = e$ to $G$ jest grupą abelową.
    \item Niech $(H, +)$ będzie grupą oraz $f : G \rightarrow H$ będzie homomorfizmem grup. Wykaż, że $\ker_f < G$ (jądro jest podgrupą $G$).
    \item Podaj przykład struktury z działaniem przemiennym, ale nie łącznym.
\end{enumerate}

W zadaniach 12--14 odnosimy się do pierścienia $(R, +, \cdot)$
\begin{enumerate}[resume]
    \item Wykaż, że dla każdego $a,b \in R$ zachodzi $a(-b) = -(ab) = (-a)b$.
    \item Wykaż, że dla każdego $a,b \in R$ zachodzi $(-a)(-b) = ab$.
    \item * Wykaż, że jeżeli $R$ jest skończonym pierścieniem całkowitym, to jest też ciałem.
    \item * Wykaż, że jedynym automorfizmem $\mathbb{Q}$ jako ciała jest identyczność.
\end{enumerate}

W zadaniach 16--21 dana jest funkcja $f : X \rightarrow Y$ oraz $A,B \subseteq X$ i $C,D \subseteq Y$.
\begin{enumerate}[resume]
    \item Wykaż, że $f(A \cup B) = f(A) \cup f(B)$.
    \item Wykaż, że $f(A \cap B) \subseteq f(A) \cap f(B)$.
    \item Wykaż, że $f(A \setminus B) \subseteq f(A) \setminus f(B)$.
    \item Wykaż, że $A \subseteq f^{-1}(f(A))$.
    \item Wykaż, że $f^{-1}(C \cup D) = f^{-1}(C) \cup f^{-1}(D)$.
    \item Wykaż, że $f^{-1}(C \cap D) = f^{-1}(C) \cap f^{-1}(D)$.
    \item Wykaż, że funkcja $f : X \rightarrow Y$ jest iniekcją wtedy i tylko wtedy gdy dla dowolnych $A, B \subseteq X$ zachodzi równość $f(A\setminus B) = f(A) \setminus f(B)$.
\end{enumerate}

\end{document}
