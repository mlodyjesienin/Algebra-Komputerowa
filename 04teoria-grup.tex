\documentclass{beamer}

\usepackage{polski}
\usepackage{pgf,tikz}
\usepackage{tgheros}       

\usetheme{BIT}

\usepackage[utf8]{inputenc}
\usepackage{amssymb,amsmath,amsthm}

\usetikzlibrary{arrows}
\usetikzlibrary{shapes,decorations}

\newcommand{\zero}{\mathbf{0}}
\newcommand{\one}{\mathbf{1}}

\title{Algebra Komputerowa}
\subtitle{Elementy Teorii Grup}
\author{Filip Zieli\'nski}
\date{2025}
 
\begin{document}

\begin{frame}
    \titlepage
\end{frame}
 
\begin{frame}{Spis Treści}
    \tableofcontents
\end{frame}

\section{Grupy Cykliczne}

\begin{frame}{Grupy Cykliczne}
    \begin{block}{Definicja}
        Grupa $(G,\cdot)$ jest \alert{grupą cykliczną} wtw. gdy, istnieje taki element $a \in G$, że każdy element grupy $G$
        jest jego potęgą, to znaczy
        $$\forall g \in G \ \exists k \in \mathbb{Z}  : \quad g = a^k .$$
        Element $a$ nazywamy wtedy \textit{generatorem grupy cyklicznej}
    \end{block}
    \begin{alertblock}{Uwaga}
        W konwencji multiplikatywnej mówimy o $k$-tej \textit{potędze elementu} $a$ i zapisujemy ją jako
        $a^k = \underbrace{a \cdot a \cdot \ldots \cdot a}_{k}$, natomiast w konwencji addytywnej, mówimy o
        $k$-tej \textit{wielokrotności elementu} $a$ i zapisujemy $k \cdot a = \underbrace{a + a + \ldots + a}_{k}$.
    \end{alertblock}

\end{frame}

\begin{frame}{Grupy Cykliczne}
    \begin{alertblock}{Konwencja}
        Jeśli $a$ jest generatorem grupy $G$ to piszemy $G = \langle a \rangle.$ 
    \end{alertblock}
    \begin{exampleblock}{Przykład}
        Grupami cyklicznymi są np.
        \begin{enumerate}
            \item $(\sqrt[n]{1}, \cdot)$
            \item $(\mathbb{Z}_n, +)$
            \item $\mathbb{Z}$ 
        \end{enumerate}
    \end{exampleblock}
\end{frame}

\begin{frame}{Abelowość grup cyklicznych}
    \begin{block}{Obserwacja}
        Każda grupa cykliczna jest abelowa.
    \end{block}
    \begin{proof}
        Rozważmy grupę $\langle a \rangle $. Wystarczy zauważyć, że z łączności wprost wynika $a^pa^q = a^qa^p.$
    \end{proof}
\end{frame}
    
\begin{frame}{Skończone grupy cykliczne}
    \begin{block}{Twierdzenie}
        Grupa cykliczna $\langle a \rangle$ jest skończona wtedy i tylko wtedy, gdy istnieją liczby całkowite $p,q$, gdzie 
        $p \neq q$, takie, że $a^p =a^q$.
    \end{block}
    \pause
    \begin{alertblock}{Obserwacja}
        Grupę cykliczną rzędu $n$ można zapisać w postaci $\{a^0,a^1 \ldots, a^{n-1}\}$, 
        natomiast nieskończoną grupę cykliczną w postaci $\{\ldots, a^{-1}, a^0, a^1, \ldots\}.$
    \end{alertblock}
    \begin{alertblock}{Wniosek}
        Grupa cykliczna $\langle a \rangle$ jest nieskończona wtedy i tylko wtedy, gdy dla każdego $p \neq q$, $p,q \in \mathbb{Z}$
        zachodzi $a^p \neq a^q$.
    \end{alertblock}
\end{frame}

\begin{frame}{izomorfizm grup cyklicznych}
    \begin{block}{Twierdzenie}
        Wszystkie grupy cykliczne nieskończonego rzędu są izomorficzne.
        
        Wszystkie grupy cykliczne skończone równych rzędów są izomorficzne.
    \end{block}
\end{frame}

\end{document}