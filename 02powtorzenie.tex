\documentclass{beamer}

\usepackage{polski}
\usepackage{pgf,tikz}
\usepackage{tgheros}       

\usetheme{BIT}

\usepackage[utf8]{inputenc}
\usepackage{amssymb,amsmath,amsthm}

\usetikzlibrary{arrows}
\usetikzlibrary{shapes,decorations}

\newcommand{\zero}{\mathbf{0}}
\newcommand{\one}{\mathbf{1}}
\newcommand{\image}{\textrm{Im}}
\let\phi\varphi

\title{Algebra Komputerowa}
\subtitle{Powtórzenie Podstaw Algebry}
\author{Filip Zieli\'nski}
\date{2025}
 
\begin{document}

\begin{frame}
    \titlepage
\end{frame}
 
\begin{frame}{Spis Treści}
    \tableofcontents
\end{frame}

\section{Działania}
\begin{frame}{Działanie Wewnętrzne}
    \begin{block}{Definicja}
        Niech $X$ będzie ustalonym niepustym zbiorem. Dwuargumentowym \textbf{działaniem wewnętrznym} na zbiorze $X$ nazywamy  dowolne odwzorowanie $h: X \times X \rightarrow X$.
        Dla elementów $x,y \in X$ wartość $h(x,y)$ nazywamy wynikiem działania $h$ na argumentach $x,y$. 
    \end{block}
    \pause 
    \begin{exampleblock}{Przykład}
        Działaniami wewnętrznymi są np. 
        \begin{itemize}
            \item  $h(x,y) = \frac{x+y}{2}$, dla $X = \mathbb{Q}$ 
            \item $h(x,y) = 2^{xy}$, dla $X = \mathbb{N}$ 
            \item $h(f,g) =  f \circ g $, dla $ X = \mathcal{F}(\mathbb{R}, \mathbb{R})$
        \end{itemize}
        Działaniem wewnętrznym \textbf{nie jest} np. 
        \begin{itemize}
            \item $h(x,y) = x + y$, dla $X = \{ a \in \mathbb{N} : 2\mid a \lor 3\mid a \}$
        \end{itemize}
        
    \end{exampleblock}
\end{frame}

\begin{frame}{Działanie Zewnętrzne}
    \begin{block}{Definicja}
        Dwuargumentowym \textbf{działaniem zewnętrznym} w niepustym zbiorze $X$ nad niepustym zbiorem $F$ nazywamy odwzorowanie
        $g : F \times X \rightarrow X$.
    \end{block}
    \pause 
    \begin{exampleblock}{Przykład}
        Działaniami zewnętrznymi są np.
        \begin{itemize}
            \item $g(\alpha, [x,y]) = [\alpha \dot x, \alpha \dot y]$, dla $F = \mathbb{R}, X = \mathbb{R}^2$
            \item $g(a,x) = nx$, dla $F = \mathbb{Z},  X =\mathbb{R}$
            \item $g(a,b) = ab$, dla $F = \mathbb{Z}, X = \{ b \in \mathbb{Z} : 3 \mid b \}$
        \end{itemize}
        Działaniem zewnętrznym \textbf{nie jest} np. 
        \begin{itemize}
            \item $g(a,b) = a +b $, dla $F = \mathbb{R}, X = \mathbb{Q}$
        \end{itemize}
    \end{exampleblock}
\end{frame}

\begin{frame}{Oznaczenia}
    \begin{alertblock}{Konwencja}
        Zwyczajowo działania oznaczamy symbolami : 
        
        $ +, \star, \cdot, \circ, \oplus, \otimes $

        Natomiast wynik działania oznaczamy odpowiednio przez: $x +y, x\star y, x \cdot y, x\circ y, x \oplus y, x\otimes y$
    \end{alertblock}
\end{frame}

\section{Grupy}
\begin{frame}{Półgrupy, Monoidy, Grupy}
    \setbeamercovered{transparent}
    \begin{block}{Definicja}
        Niepusty zbiór $G$ z działaniem wewnętrznym $\oplus$ nazywamy \only<1>{\textbf{Półgrupą}} \only<2>{\textbf{Monoidem}} \only<3>{\textbf{Grupą}} \only<4,5>{\textbf{Grupą abelową}}
        jeżeli spełnione są następujące warunki 
        \begin{enumerate}
            \onslide<1->{\item $\forall x,y,z \in G \quad (x \oplus y) \oplus z = x \oplus (y \oplus z)$ \hfill \textit{(łączność)} }
            \onslide<2->{ \item $\exists e \in G : \forall x \in G   \quad x \oplus e = e \oplus x = x$ \hfill \textit{(el. neutralny)}}
            \onslide<3->{ \item $\forall x \in G \: \exists x' \in G \quad x \oplus x' = x' \oplus x = e $ \hfill \textit{(el. odwrotne)}}
            \onslide<4->{ \item $\forall x,y \in G \quad x \oplus y = y \oplus x$ \hfill \textit{(przemienność)}}
        \end{enumerate}
    \end{block} 
    \setbeamercovered{}
    \onslide<5->{
        \begin{alertblock}{Konwencja addytywna}
            Element neutralny grupy $G$ oznaczamy  często jako $\zero$. W szczególności jeśli mowa o "dodawaniu", oznaczanym przez $+ , \oplus$
            Elementy symetryczne nazywamy "przeciwnymi" i oznaczamy $-a$. Zapis $a-b$ należy rozumieć jako $a + (-b)$
        \end{alertblock}
    }
\end{frame}


\begin{frame}{Półgrupy, Monoidy, Grupy}
    \begin{exampleblock}{Przykład}
        Półgrupą jest np.
        \begin{itemize}
            \item $(\mathbb{N}\setminus \{0\} , +)$
        \end{itemize}
        Monoidem jest np. 
        \begin{itemize}
            \item $ ( \mathbb{Z}, \cdot )$
        \end{itemize}
        Grupą jest np.
        \begin{itemize}
            \item $(\textit{Bij}(\mathbb{R},\mathbb{R}), \circ)$ - Zbiór funkcji bijektywnych z $\mathbb{R}$ w $\mathbb{R}$ ze składaniem odwzorowań.
        \end{itemize}
        Grupą Abelową jest np.
        \begin{itemize}
            \item $( \mathbb{Z}, +)$
        \end{itemize}
        
    \end{exampleblock}
    \pause 
    \begin{alertblock}{Konwencja}
        Jeżeli działanie w grupie wynika z kontekstu, możemy je pomijać w zapisie i utożsamiać grupę ze zbiorem.
    \end{alertblock}
\end{frame}

\begin{frame}{Podgrupa}
    \begin{block}{Definicja}
        Niech $(G, +)$ będzie grupą. Niepusty podzbiór $H \subseteq G$ nazywamy podgrupą grupy $G$
        jeżeli zachodzi warunek 
        \begin{align*}
            & \forall x,y \in H  \quad x - y \in H
        \end{align*}
    \end{block}
    \pause 
    \begin{exampleblock}{Przykład}
        Podgrupami  $(\mathbb{Z}, +)$ są np.

        \begin{tabular}{cc}
              $2\mathbb{Z} = \{a \in \mathbb{Z} : 2 \mid a\}$, & 
            $\{ \zero\}$
        \end{tabular}

        Podgrupami  $(\mathbb{R}\setminus \{0\}, \cdot)$ są np.
        
        \begin{tabular}{cc}
            $\mathbb{Q}\setminus \{0\}$, & $\mathbb{R}\setminus \{0\}$
        \end{tabular}
    \end{exampleblock}
    \pause 
    \begin{alertblock}{Konwencja}
        oznaczenie $H < G$ należy rozumieć jako \textit{"H jest pogrupą G"}.
    \end{alertblock}
\end{frame}

\begin{frame}{Homomorfizmy Grup}
    \begin{block}{Definicja}
        Homomorfizmem między grupą $(G, \oplus)$ oraz grupą $(H, \otimes)$ nazywamy dowolne odwzorowanie 
        $h : G \rightarrow H$, spełniające warunek
        $$ \forall x,y \in G \quad h(x \oplus y) = h(x) \otimes h(y)$$
    \end{block}
    \pause 
    \begin{exampleblock}{Przykład}
        Homomorfizmem grup $(\mathbb{Z}, +), (\mathbb{R}, \cdot)$ jest np.
        \begin{itemize}
            \item $h(x) = e^x$
        \end{itemize} 
        Homomorfizmem grup $(\mathbb{Z}, +), (\mathbb{Z}, +)$ jest np. 
        \begin{itemize}
            \item $h(x) = 2x$ 
        \end{itemize}
    \end{exampleblock}
\end{frame}

\section{Pierścienie}

\begin{frame}{Pierścienie}
    \begin{block}{Definicja}
        Zbiór $R$ z dwoma działaniami $\oplus, \otimes$ nazywamy \only<1>{\alert{Pierścieniem}} \only<2>{\alert{Pierścieniem z jedynką}} \only<3>{\alert{Pierścieniem przemiennym z jedynką}}, jeżeli zachodzą następujące warunki
        \begin{enumerate}
            \item $(R, \oplus)$ jest grupą abelową 
            \item $(R, \otimes)$ jest \only<1>{półgrupą} \only<2>{monoidem} \only<3>{monoidem przemiennym}
            \item $\forall x,y,z \in R \quad x \otimes (y \oplus z) = x \otimes y \oplus x \otimes z \land $ \\ $(x \oplus y) \otimes z = x \otimes z \oplus y \otimes z$ \hfill \textit{(rozdzielność mn. wzg. dod.)} 
        \end{enumerate}
    \end{block}
\end{frame}

\begin{frame}{Pierścienie}
    \begin{exampleblock}{Przykład}
        Pierścieniem jest np.
        \begin{itemize}
            \item $(5\mathbb{Z}, +, \cdot)$
        \end{itemize}
        Pierścieniem z jedynką jest np.
        \begin{itemize}
            \item $(\mathcal{M}_2(\mathbb{R}), +, \cdot)$
        \end{itemize}
        Pierścieniami przemiennymi z jedynką są np.
        \begin{itemize}
            \item $(\mathbb{Z}, +, \cdot )$
            \item $(\mathbb{R}[x], +, \cdot )$
            \item $(\mathbb{Z}_n, +, \cdot )$
            \item $(\mathbb{R}[x_1,\ldots, x_n], +, \cdot )$
        \end{itemize}
    \end{exampleblock}
\end{frame}

\begin{frame}{Pierścienie}
    \begin{block}{Obserwacja}
        Dla dowolnego pierścienia $(R, \oplus, \otimes)$ zachodzi:
        $$ \forall x \in R \quad x \otimes \zero =  \zero \otimes x = \zero$$
    \end{block}
    \pause 
    \begin{proof}
        Przeprowadzimy dowód, że $x \otimes \zero = 0$.
        Załóżmy nie wprost, że istnieje $x \in R$ takie, że $x \otimes \zero = y, y\neq \zero$.
        Możemy zapisać $y = x \otimes \zero = x \otimes (\zero \oplus \zero) = x \otimes \zero \oplus x \otimes \zero = y \oplus y$.
        Dostaliśmy zatem $y = y \oplus y$ co po obustronnym dodaniu $-y$ prowadzi do $y = \zero$ co jest sprzeczne z założeniem.
        Dowód faktu, że $\zero \otimes x = \zero$ można przeprowadzić analogicznie.
    \end{proof}
\end{frame}

\begin{frame}{Pierścienie}
    \begin{alertblock}{Konwencja}
        Jeżeli $(R, \oplus, \otimes)$ jest pierścieniem, to zwyczajowo działanie $\oplus$ nazywamy dodawaniem, a $\otimes$ mnożeniem.
        Dodatkowo, jeżeli $(R, \otimes)$ jest monoidem, to jego element neutralny nazywamy "jedynką" i oznaczamy $\one$. 
    \end{alertblock}
\end{frame}

\begin{frame}{Podpierścienie}
    \begin{block}{Definicja}
        Niepusty podzbiór $S$ pierścienia $(R, \oplus, \ominus)$ nazywamy podpierścieniem $R$, jeżeli $(S,\oplus)$ jest podgrupą (addytywną) $(R,\oplus)$ oraz zbiór $S$ jest zamknięty ze względu na mnożenie.
        Dodatkowo, jeżeli, $R$ jest pierścieniem z jedynką dodaje się warunek $\one \in S$. 
    \end{block}
    \begin{exampleblock}{Przykład}
        Podpierścieniem pierścienia $\mathbb{R}$ są np.
        \begin{itemize}
            \item $\mathbb{Q}$
            \item $\mathbb{Z}$
        \end{itemize}
    \end{exampleblock}
\end{frame}

\begin{frame}{Pierścienie z Dzielnikami Zera}
    \begin{block}{Definicja}
        Niech $(R, \oplus, \otimes)$ będzie pierścieniem. Wtedy $a,b \in R, a,b \neq 0$ są \textbf{dzielnikami zera} wtedy i tylko wtedy gdy $a \otimes b = 0$.
    \end{block}
    \begin{block}{Definicja}
        Pierścień, w którym nie występują dzielniki zera, nazywamy \textbf{Pierścieniem całkowitym}.
    \end{block}
    \begin{exampleblock}{Przykład}
        \begin{enumerate}
            \item W pierścieniu $\mathbb{Z}_6$ elementy $2,3$ są dzielnikami zera, ponieważ $2,3 \neq \zero \land 2\cdot 3 = 6 = \zero$.
            \item Pierścień $\mathbb{Z}$ jest pierścieniem całkowitym. 
        \end{enumerate}        
    \end{exampleblock}
\end{frame}

\begin{frame}{Homomorfizmy Pierścieni}
    \begin{block}{Definicja}
        Niech $(R, +, \cdot)$ oraz $(S, \oplus, \otimes)$ będą dowolnymi pierścieniami.
        Homomorfizmem pierścieni $R,S$ nazywamy dowolne odwzorowanie $h : R \rightarrow S$ takie, że:
        \begin{align*}
            \forall a,b \in R \quad h(a + b) & = h(a) \oplus h(b) \\
            \forall a,b \in R \quad h(a \cdot b) & = h(a) \otimes h(b) 
        \end{align*}
        \uncover<2->{Dodatkowo, jeśli, R,S są pierścieniami z jedynką, musi zachodzić 
        $$ h(\one_R) = \one_S$$}
    \end{block}
\end{frame}

\begin{frame}{Homomorfizmy Pierścieni}
    \begin{exampleblock}{Przykład}
        Homomorfizmem pierścieni $\mathbb{Z}, \mathbb{Z}[x]$ jest np.
        \begin{itemize}
            \item $h(a) = a$
        \end{itemize}
        Homomorfizmem pierścieni $\mathbb{Z}, \mathbb{Z}_n$ jest np.
        \begin{itemize}
            \item $h(a) = a $ (mod $n$)
        \end{itemize}
    \end{exampleblock}
\end{frame}

\section{Ciała}

\begin{frame}
    \begin{block}{Definicja}
        Pierścień z jednością $(K, \oplus, \otimes)$ nazywamy ciałem, jeżeli $(K\setminus \{\zero\}, \otimes)$ jest grupą abelową.
    \end{block}
    \begin{exampleblock}{Przykład}
        Ciałami są np.
        \begin{itemize}
            \item $\mathbb{R}$
            \item $\mathbb{Q}$
            \item $\mathbb{C}$
            \item $\mathbb{Z}_p$, dla $p$ będącego liczbą pierwszą
        \end{itemize}
    \end{exampleblock}
    \begin{alertblock}{Konwencja}
        Elementy symetryczne w działaniu "mnożenia" nazywamy elementami odwrotnymi i oznaczamy $a^{-1}$
    \end{alertblock}
\end{frame}

\begin{frame}{Ciała}
    \begin{block}{Obserwacja}
        Dowolne ciało $(K, \oplus, \otimes)$ jest pierścieniem całkowitym.
    \end{block}
    \pause 
    \begin{proof}
        Załóżmy nie wprost, że istnieją $ a,b \in K, a,b \neq \mathbf{0 }$ takie, że $ a \otimes b = \zero $. z tego wynika, że
        $a^{-1} \otimes a \otimes b = a^{-1} \otimes \zero $ z czego wynika $ \one \otimes b = \zero $ co jest równoważne z $b = \zero$, co jest sprzecze z założeniem.    
    \end{proof}
\end{frame}

\begin{frame}{Podciała}
    \begin{block}{Definicja}
        Niech $K$ będzie ciałem. Niepusty podzbiór $L$ zbioru $K$ nazywamy podciałem, gdy $L$ jest podpierścieniem $K$ oraz zachodzi
        $$\forall a \in L\setminus \{\zero\} \quad a^{-1} \in L$$
    \end{block}
    \begin{exampleblock}{Przykład}
        Podciałem $\mathbb{C}$ są np.
        \begin{itemize}
            \item $\mathbb{R}$
            \item $\mathbb{Q}$
            \item $\mathbb{Q}[\sqrt{2}] = \{ a + b\sqrt{2} : a,b \in \mathbb{Q} \}$
        \end{itemize}
    \end{exampleblock}
\end{frame}

\begin{frame}{Homomorfizmy Ciał}
    \begin{block}{Definicja}
        Niech $(R, + , \cdot)$ oraz $(S, \oplus, \otimes)$ będą dowolnymi ciałami. Homomorfizem ciał $R$ i $S$ nazywamy dowolne odwzorowanie $h : R \rightarrow S$ spełniające 
        \begin{align*}
            \forall a,b \in R \quad h(a + b) & = h(a) \oplus h(b) \\
            \forall a,b \in R \quad h(a \cdot b) & = h(a) \otimes h(b) 
        \end{align*}
    \end{block}
\end{frame}
\section{Funkcje}

\begin{frame}{Szczególne homomorfizmy}
    \begin{block}{Definicja}
        Homomorfizm (grup, pierścieni, ciał) nazwiemy 
        \begin{itemize}
            \item \textit{monomorfizmem}, gdy jest iniektywny
            \item \textit{epimorfizmem}, gdy jest surjektywny
            \item \textit{izomorfizmem}, gdy jest bijektywny
            \item \textit{endomorfizmem}, gdy dziedzina jest równa przeciwdziedzinie
            \item \textit{automorifzmem}, gdy jest to endomorfizm bijektywny 
        \end{itemize}
    \end{block}
\end{frame}

\begin{frame}{Obraz i Przeciwobraz zbioru}
    Niech $f : X \rightarrow Y$ będzie dowolnym odwzorowaniem z $X$ do $Y$. 
    \begin{block}{Definicja}
        \alert{Obrazem zbioru $A$} $ \subseteq X$ przez odwzorowanie $f$ nazywamy zbiór 
        $\{y \in Y \mid \exists x \in A : f(x) = y\}$ i oznaczamy przez $f(A)$.
    \end{block}
    \begin{block}{Definicja}
        \alert{Przeciwobrazem zbioru $B$ } $ \subseteq Y$ przez odwzorowanie $f$ nazywamy zbiór 
        $\{ x \in X  \mid  f(x) \in B \}$ i oznaczamy przez $f^{-1}(B)$.
    \end{block}
\end{frame}

\begin{frame}{Obraz i Jądro odwzorowania }
    Niech $f : X \rightarrow Y$ będzie dowolnym odwzorowaniem z $X$ do $Y$. 
    \begin{block}{Definicja}
        \alert{Obrazem odwzorowania $f$} nazywamy zbiór $f(X)$ i oznaczamy przez $\image_f$.
    \end{block}
    W przypadku gdy przeciwdziedzina dziedziny tworzy strukturę z elementem neutralnym oznaczanym przez $\zero$
    definiujemy dodatkowo \textit{jądro odwzorowania}
    \begin{block}{Definicja}
        \alert{Jądrem odwzorowania $f$} nazywamy zbiór $f^{-1}(\zero)$ i oznaczamy przez $\ker_f$.
    \end{block}
\end{frame}

\begin{frame}{Obraz i Jądro odwzorowania}
    \begin{exampleblock}{Przykład}
        Niech $f : \mathbb{Z} \rightarrow \mathbb{Z}_7$ będzie zadane wzorem $f(a) = a $ (mod $7$).
        Wtedy
        \begin{itemize}
            \item $f(\{1,9,15\}) = \{1,2\}$
            \item $f^{-1}(1) = \{ 7k + 1 \mid k \in \mathbb{Z}\}$
            \item $\image_f = f(\mathbb{Z}) = \mathbb{Z}_7$
            \item $\ker_f =f^{-1}(0) = \{ 7k \mid k \in \mathbb{Z} \}$ 
        \end{itemize}
    \end{exampleblock}
\end{frame}

\section{Iloczyn Kartezjański i Suma Prosta}
\begin{frame}{Iloczyn Kartezjański}
    Niech $A,B$ będą dowolnymi niepustymi zbiorami.
    \begin{block}{Definicja}
         \alert{Iloczynem Kartezjańskim} $A,B$ nazywamy zbiór
        $\{(a,b) \mid a \in A \land b \in B\}$ i oznaczamy przez $A \times B$.
    \end{block}
    \begin{block}{Definicja}
        $n$-tą potęgą zbioru $A$ rozumiemy jako $ \underbrace{A \times A \times \ldots \times A}_{n} 
        = \{(a_1,\ldots, a_n) \mid a_1,\ldots,a_n \in A \}$ i oznaczamy  przez $A^n$.
    \end{block}
\end{frame}

\begin{frame}{Suma Algebraiczna, Prosta}
    Niech $(G, +)$ będzie grupą oraz niech $A,B$ będą dowolnymi niepustymi podzbiorami $G$.
    \begin{block}{Definicja}
        \alert{Sumą Algebraiczną} zbiorów $A,B$ nazywamy zbiór $\{ a + b \mid a \in A \land b \in B\}$
        i oznaczamy przez $A + B$.
    \end{block}
    Jeżeli zachodzi własność, że dla każdego $c \in A +B $ istnieje dokładnie jedna para $a,b$ taka, że $a \in A, b \in B$ oraz $c = a +b$, 
    to mówimy o \alert{Sumie Prostej} zbiorów $A,B$. Zwyczajowo, sumę prostą zbiorów $A,B$ oznaczamy przez $A \oplus B$. 
    \pause 
    Zauważmy, że istnieje naturalny izomorfizm  $ \phi : A \times B \rightarrow A \oplus B$ zadany przez
    $\phi(a,b) = a + b$. Z tego powodu, często w literaturze suma prosta (wewnętrzna) jest nierozróżnialna z iloczynem kartezjańskim. 
\end{frame}

\begin{frame}
        \centering 
        \LARGE Pytania, wątpliwości, uwagi ? 
\end{frame}

\end{document}