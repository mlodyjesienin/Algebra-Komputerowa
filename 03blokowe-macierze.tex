\documentclass{beamer}

\usepackage{polski}
\usepackage{pgf,tikz}
\usepackage{tgheros}       

\usetheme{BIT}

\usepackage[utf8]{inputenc}
\usepackage{amssymb,amsmath,amsthm}
\usepackage{mathtools}

\usetikzlibrary{arrows}
\usetikzlibrary{shapes,decorations}

\newcommand{\zero}{\mathbf{0}}
\newcommand{\one}{\mathbf{1}}
\newcommand{\id}{\mathcal{I}}
\renewcommand{\det}{\textrm{det}}
\title{Algebra Komputerowa}
\subtitle{Macierze Blokowe}
\author{Filip Zieli\'nski}
\date{2025}
 
\begin{document}

\begin{frame}
\titlepage
\end{frame}
 
\begin{frame}{Spis Treści}
    \tableofcontents
\end{frame}

\section{Wstęp} 

\begin{frame}{Macierze Blokowe}
    W poniższych rozważaniach skupimy się na macierzach blokowych $2 \times 2$. Wiele wyników da się jednak uogólnić.
    \begin{block}{Definicja}
        Rozważmy macierze $A \in \mathcal{M}_{n_1 \times m_1}, B \in \mathcal{M}_{n_1 \times m_2}, C \in \mathcal{M}_{n_2 \times m_1}, D \in \mathcal{M}_{n_2 \times m_2}$ nad ustalonym ciałem $K$.
        Wtedy, \alert{Macierzą Blokową} (klatkową) $X \in \mathcal{M}_{n_1 + n_2 \times m_1 + m_2}$ złożoną z  $A,B,C,D$ definiujemy jako $X = (x_{ij})$, gdzie 
        $$x_{ij} = \begin{cases}
            a_{ij} & i \leq n_1 , \quad  j \leq m_1 \\ 
            b_{(i)(j-m_1)} &  i \leq n_1 , \quad m_1 < j \leq m_1 + m_2 \\
            c_{(i-n_1)(j)} & n_1 < i \leq n_1 + n_2, \quad j \leq m_1 \\
            d_{(i-n_1)(j-m_1)} & n_1 < i \leq n_1 + n_2 \quad  m_1 < j \leq m_1 + m_2 \\
        \end{cases}$$
    \end{block}    
\end{frame}

\begin{frame}{Macierze Blokowe}
    Takie macierze zapisujemy jako
    $$ X = \begin{bmatrix}
        A & B \\
        C & D
    \end{bmatrix}$$
    Zauważmy, że macierze można dzielić na bloki na wiele sposobów, 
    $$X = \left[ \begin{array}{ccc|c}
        1 & 2 & 3 & 4 \\ \hline 
        5 & 6 & 7 & 8 \\ 
        9 & 10 & 11 & 12 \\ 
        13 & 14 & 15 & 16 \\
    \end{array}\right] =  \left[ \begin{array}{cc|cc}
        1 & 2 & 3 & 4 \\ 
        5 & 6 & 7 & 8 \\ \hline 
        9 & 10 & 11 & 12 \\ 
        13 & 14 & 15 & 16 \\
    \end{array}\right] $$
    ale zawsze zachodzi $colA = colC, \quad colB = colD,\quad  rowA = rowB, \quad rowC = rowD$
\end{frame}

\begin{frame}{Szczególne Macierze Blokowe}
    \begin{alertblock}{Konwencja}
        Blok, będący macierzą dowolnych wymiarów wypełnioną samymi zerami oznaczamy jako $\zero$.
    \end{alertblock}
    \begin{block}{Definicja}
        \alert{Diagonalną (Przekątniową) macierzą blokową} nazywamy macierz blokową $D$, jeśli da się ją zapisać jako
        $$ D = \begin{bmatrix}
            D_1 & \zero \\ \zero & D_2 \\
        \end{bmatrix}$$
    \end{block}
    
\end{frame}

\begin{frame}{Szczególne Macierze Blokowe}
    \begin{block}{Definicja}
        \alert{Trójkątno-górną macierzą blokową} nazywamy macierz blokową $U$, jeśli da się ją zapisać jako
        $$ U = \begin{bmatrix}
            U_{11} & U_{12} \\ \zero & U_{22} \\  
        \end{bmatrix}$$
    \end{block}
    \begin{block}{Definicja}
        \alert{Trójkątno-dolną macierzą blokową} nazywamy macierz blokową $L$, jeśli da się ją zapisać jako 
        $$ L = \begin{bmatrix}
            L_{11} & \zero \\ L_{21} & L_{22} \\
        \end{bmatrix}$$
    \end{block}
\end{frame}

\begin{frame}{Szczególne Macierze Blokowe}
    \begin{alertblock}{Konwencja}
        W notacji macierzy blokowej $\id$ może oznaczać macierz jednostkową dowolnego wymiaru. 
    \end{alertblock}
    \begin{block}{Uwaga }
        Macierz jednostkową można zapisać w postaci blokowej jako
        $$ \id = \begin{bmatrix}
            \id & \zero \\ \zero & \id \\
        \end{bmatrix}$$
    \end{block}
\end{frame}

\section{Podstawowe Operacje} 

\begin{frame}
    \begin{block}{Obserwacja}
        Niech $X,Y$ będą macierzami tych samych wymiarów. Jeżeli $X,Y$ podzielimy na bloki 
        odpowiednio \textit{tych samych wymiarów}, to macierze blokowe można dodawać
        $$X = \begin{bmatrix}
            A_1 & B_1 \\ C_1 & D_1 \\
        \end{bmatrix},
        Y = \begin{bmatrix}
            A_2& B_2 \\ C_2 & D_2 \\
        \end{bmatrix}, \quad 
        X + Y = \begin{bmatrix}
            A_1 + A_2 & B_1 + B_2 \\ C_1 + C_2 & D_1 + D_2 \\ 
        \end{bmatrix}$$
    \end{block}
    Równie naturalnie, zdefiniowane jest odejmowanie macierzy blokowych jak i mnożenie macierzy przez skalar z ciała.
\end{frame}

\begin{frame}{Transpozycja Macierzy Blokowych}
    \begin{block}{Twierdzenie}
        Niech $X = \begin{bmatrix}
            A & B \\ C & D \\
        \end{bmatrix}$
        będzie macierzą blokową. Transpozycja macierzy $A$ jest macierzą blokową, zadaną jako $$A^T = \begin{bmatrix}
            A^T & C^T \\ B^T & D^T \\   
        \end{bmatrix} $$
    \end{block}
\end{frame}

\section{Mnożenie}

\begin{frame}{Mnożenie Macierzy Blokowych}
    \begin{block}{Twierdzenie}
        Niech, $X \in \mathcal{M}_{n_1 + n_2 \times m_1 + m_2},Y \in \mathcal{M}_{m_1 + m_2 \times p_1 + p_2}$ będą macierzami nad tym samym ciałem $K$ podzielonymi na bloki w następujący sposób
        $$X =\begin{bmatrix}
            A_1 & B_1 \\ C_1 & D_1 \\
        \end{bmatrix}, \quad Y = \begin{bmatrix}
            A_2 & B_2 \\ C_2 & D_2 \\ 
        \end{bmatrix},$$  gdzie $col A_1 = col C_1 = row A_2 = row B_2$ oraz $col B_1 = col D_1 = row C_2 = row D_2 .$
        \newline 
        Wtedy 
        $$ M = XY = \begin{bmatrix}
            A_1 A_2 +  B_1 C_2 & A_1 B_2 + B_1 D_2 \\
            C_1 A_2 + D_1 C_2 & C_1 B_2 + D_1 D_2 \\
        \end{bmatrix}$$
    \end{block}
\end{frame}

\section{Eliminacja Gaussa}

\begin{frame}{Operacje Elementarne - Mnożenie}
    Niech $M = \begin{bmatrix}
        A & B \\ C & D \\ 
    \end{bmatrix}$ będzie macierzą blokową o wymiarach: \\
    $A$ \textendash \ $n \times m$, \quad$B$ \textendash \ $n \times k$,\quad $C$ \textendash \ $p \times m$, \quad $D$ \textendash \ $p \times k$ \\
    oraz niech $E$ będzie dowolną macierzą o wymiarach $n \times n$. 
    \begin{alertblock}{Obserwacja}
        Operacje $\begin{bmatrix}
            A & B \\ C & D \\
        \end{bmatrix} \xrightarrow{R_1 = E \cdot R_1} \begin{bmatrix}
            E  A & E  B \\ C & D \\ 
        \end{bmatrix}$ da się wykonać i da się przedstawić jako złożenie operacji elementarnych na wierszach.
    \end{alertblock}
\end{frame}

\begin{frame}{Operacje Elementarne - Odejmowanie}
    Niech $M = \begin{bmatrix}
        A & B \\ C & D \\ 
    \end{bmatrix}$ będzie macierzą blokową o wymiarach: \\
    $A$ \textendash \ $n \times m$, \quad$B$ \textendash \ $n \times k$,\quad $C$ \textendash \ $p \times m$, \quad $D$ \textendash \ $p \times k$ \\
    oraz niech $F$ będzie dowolną macierzą o wymiarach $n \times p$
    \begin{alertblock}{Obserwacja}
        Operacje $\begin{bmatrix}
            A & B \\ C & D \\
        \end{bmatrix} \xrightarrow{R_1 = R_1 -  F \cdot R_2} \begin{bmatrix}
            A - F  C & B - F D \\ C & D \\ 
        \end{bmatrix}$ da się wykonać i da się przedstawić jako złożenie operacji elementarnych na wierszach.
    \end{alertblock}
\end{frame}
\section{Odwrotności}

\begin{frame}{Macierze blokowo-diagonalne}
    \begin{block}{Twierdzenie}
        Niech $D = \begin{bmatrix}
            D_1 & \zero \\ \zero & D_2 \\ 
        \end{bmatrix}$
        będzie kwadratową macierzą blokowo-diagonalną, gdzie $D_1, D_2$ są blokami kwadratowymi. Macierz $D$ jest nieosobliwa wtedy i tylko wtedy gdy macierze 
        $D_1$ oraz $D_2$ są nieosobliwe oraz $D^{-1}$ zadane jest wzorem
        $$ D^{-1} = \begin{bmatrix}
            D_{1}^{-1} & \zero \\ \zero & D_{2}^{-1}
        \end{bmatrix}$$
    \end{block}
    \pause 
    \begin{proof}
        Wzór najprościej sprawdzić z definicji macierzy odwrotnej. 
        Wkw na istnienie wynika z analizy liczby liniowo niezależnych wierszy lub macierzy. 
    \end{proof}
\end{frame}

\begin{frame}{Macierze blokowo-trójkątne}
    \begin{block}{Twierdzenie}
        Niech $U = \begin{bmatrix}
            U_{11} & U_{12} \\ \zero & U_{22}
        \end{bmatrix}$
        będzie kwadratową macierzą blokową trójkątną górną, gdzie $U_{11}, U_{22}$ są blokami kwadratowymi. Macierz $U$ jest nieosobliwa wtedy i tylko wtedy 
        gdy macierze $U_{11}$ oraz  $U_{22}$ są nieosobliwe oraz $U^{-1}$ zadane jest wzorem
        $$ U^{-1} = \begin{bmatrix}
            U_{11}^{-1} & -U_{11}^{-1} U_{12} U _{22}^{-1} \\ \zero & U_{22}^{-1}
        \end{bmatrix}$$
    \end{block}
    \pause 
    \begin{proof}
        Wzór można wyprowadzić z metody Gaussa, 
        natomiast wkw wynika z analizy liczby liniowo niezależnych kolumn i wierszy.
    \end{proof}
\end{frame}

\begin{frame}{Macierze blokowo-trójkątne}
    \begin{block}{Twierdzenie}
        Niech $L = \begin{bmatrix}
            L_{11} & \zero \\ L_{21} & L_{22}
        \end{bmatrix}$
        będzie kwadratową macierzą blokową trójkątną dolną, gdzie $L_{11}, L_{22}$ są blokami kwadratowymi. Macierz $L$ jest nieosobliwa wtedy i tylko wtedy 
        gdy macierze $L_{11}$ oraz  $L_{22}$ są nieosobliwe oraz $L^{-1}$ zadane jest wzorem
        $$ L^{-1} = \begin{bmatrix}
            L_{11}^{-1} & \zero\\ -L_{22}^{-1} L_{21} L _{11}^{-1}  & L_{22}^{-1}
        \end{bmatrix}$$
    \end{block}
    \pause 
    \begin{proof}
        Wzór można wyprowadzić z metody Gaussa, 
        natomiast wkw wynika z analizy liczby liniowo niezależnych kolumn i wierszy.
    \end{proof}
\end{frame}

\begin{frame}{Ogólne Macierze}
    Niech $M = \begin{bmatrix}
        A & B \\ C & D \\ 
    \end{bmatrix}$ będzie kwadratową macierzą blokową, gdzie $A,D$ są kwadratowymi blokami. 
    \begin{block}{Definicja}
        Jeżeli $D$ jest nieosobliwe, \alert{Dopełnienie Schura} macierzy $M$ względem $D$ defniujemy jako 
        $M/D = A -BD^{-1}C$
    \end{block}
    \begin{block}{Definicja}
        Jeżeli $A$ jest nieosobliwe, \alert{Dopełnienie Schura} macierzy $M$ względem $A$ defniujemy jako 
        $M/A = D -CA^{-1}B$
    \end{block}
\end{frame}

\begin{frame}{Ogólne Macierze}
    Niech $M = \begin{bmatrix}
        A & B \\ C & D \\ 
    \end{bmatrix}$ będzie kwadratową macierzą blokową, gdzie $A,D$ są kwadratowymi blokami.
    \begin{block}{Twierdzenie}
        Niech $A$ będzie nieosobliwe. Wtedy $M^{-1}$ istnieje wtedy i tylko wtedy gdy $M/A$ jest odwracalne 
        oraz zachodzi wzór
        $$ M^{-1}  = \begin{bmatrix}
            A^{-1} + A^{-1} B (M/A)^{-1} C A^{-1} & - A^{-1} B (M/A)^{-1} \\
            -(M/A)^{-1} C A^{-1} & (M/A)^{-1} \\  
        \end{bmatrix}$$
    \end{block}
    \begin{proof}
        Wzór najprościej wyprowadzić z metody Gaussa. 
    \end{proof}
\end{frame}

\begin{frame}{Ogólne Macierze}
    Niech $M = \begin{bmatrix}
        A & B \\ C & D \\ 
    \end{bmatrix}$ będzie kwadratową macierzą blokową, gdzie $A,D$ są kwadratowymi blokami.
    \begin{block}{Twierdzenie}
        Niech $D$ będzie nieosobliwe. Wtedy $M^{-1}$ istnieje wtedy i tylko wtedy gdy $M/D$ jest odwracalne 
        oraz zachodzi wzór
        $$ M^{-1}  = \begin{bmatrix}
            (M/D)^{-1} & -(M/D)^{-1} B D^{-1} \\
            -D^{-1} C (M/D)^{-1}  & D^{-1} + D^{-1} C (M/D)^{-1} B D^{-1}
        \end{bmatrix}$$
    \end{block}
    \begin{proof}
        Wzór najprościej wyprowadzić z metody Gaussa. 
    \end{proof}
\end{frame}

\begin{frame}{Ogólne Macierze}
    \begin{alertblock}{Konwencja}
        Oznaczmy przez $\mathcal{J}$ macierz kwadratową dowolnego rozmiaru, która ma $1$ na odwrotnej przekątnej, a pozostałe 
        elementy są równe $0$. Zauważmy, że można zapisać taką macierz w postaci blokowej jako
        $$ \mathcal{J} = \begin{bmatrix}
            \zero & \mathcal{J} \\ 
            \mathcal{J} & \zero \\ 
        \end{bmatrix}$$
    \end{alertblock}
\end{frame}

\begin{frame}{Ogólne Macierze}
    \begin{block}{Obserwacja}
        Niech $M = \begin{bmatrix}
            A & B \\ C & D \\
        \end{bmatrix}$ będzie kwadratową macierzą blokową, której bloki $B, C$ są blokami kwadratowymi. 
        Zauważmy, że macierz $M\mathcal{J} = \begin{bmatrix}
            B\mathcal{J} & A\mathcal{J} \\ D\mathcal{J} & C\mathcal{J} \\
        \end{bmatrix}$ jest macierzą kwadratową o blokach kwadratowych na głównej przekątnej, zatem można policzyć jej odwrotność 
        przy pomocy wspomnianych wcześniej twierdzeń. 
    \end{block}
    \begin{block}{Obserwacja}
        Macierz odwrotną do macierzy blokowej $M$ posiadającej bloki kwadratowe na przekątnej odwrotnej, możemy wyliczyć z zależności
        $$M^{-1} = \mathcal{J}\mathcal{J}^{-1} M^{-1} = \mathcal{J} (M\mathcal{J})^{-1}$$
    \end{block}
\end{frame}

\section{Wyznacznik}

\begin{frame}{Wyznacznik Macierzy Blokowej}
    \begin{block}{Twierdzenie}
        Niech $\mathbb{K}$ będzie ciałem i niech $R$ będzie podpierścieniem przemiennym pierścienia $\mathcal{M}_{n \times n}(\mathbb{K})$ macierzy kwadratowych wymiaru $n$ o współczynnikach z ciała $\mathbb{K}$.
        Rozważmy macierz $M \in \mathcal{M}_{p \times p}(R)$. Zachodzi 
        $$ \det_{\mathbb{K}}M = \det _{\mathbb{K}}(\det_R M)$$ 
    \end{block}
    \pause 
    Twierdzenie to ma charakter ogólny 
    - jest eleganackie algebraiczne i praktycznie bezużyteczne, bo posiada pewne ewidentne ograniczenia.

\end{frame}

\begin{frame}{Macierze blokowo-diagonalne $2 \times 2$}
    \begin{block}{Twierdzenie}
        Niech będzie dana macierz blokowo-diagonalna $M = \begin{bmatrix}
            D_1 & \zero \\ \zero& D_2 \\ 
        \end{bmatrix}$. Zachodzi 
        $$ \det M = \det D_1 \cdot \det D_2 .$$
         Dodatkowo, jeśli oba bloki mają ten sam wymiar zachodzi $$ \det M = \det D_1 \cdot \det D_2 =  \det(D_1  D_2).$$
    \end{block}
    \pause
    \begin{proof}
        Należy skorzystać z indukcji względem wymiaru macierzy $D_1$ oraz rozwinięcia Laplace'a.
    \end{proof}
\end{frame}

\begin{frame}{Macierze blokowo-trójkątne}
    \begin{block}{Twierdzenie}
        Niech będzie dana kwadratowa macierz blokowo-górnotrójkątna $M = \begin{bmatrix}
            U_{11} & U_{12} \\  \zero & U_{22} \\ 
        \end{bmatrix}$ gdzie bloki $U_{11}$ oraz $U_{22}$ są kwadratowe. Wtedy zachodzi 
        $$\det M = \det U_{11} \cdot \det U_{22}.$$ 
        Dodatkowo, jeśli ich wymiary są takie same, zachodzi 
        $$\det M = \det U_{11} \cdot \det U_{22} = \det (U_{11} U_{22}).$$
    \end{block}
    \pause
    \begin{proof}
        Należy skorzystać z indukcji względem wymiaru macierzy $U_{11}$ oraz rozwinięcia Laplace'a.
    \end{proof}
\end{frame}

\begin{frame}{Macierze z blokiem zerowym}
    \begin{block}{Twierdzenie}
        Niech będzie dana kwadratowa macierz blokowa $M = \begin{bmatrix}
            A & B \\ C & D \\  
        \end{bmatrix}$
        o wszystkich blokach tego samego rozmiaru. Jeżeli choć jeden blok jest blokiem zerowym, zachodzi
        $$\det M = \det (AD - BC)$$
    \end{block}
    \pause 
    \begin{proof}
        W przypadku, gdy $C= \zero$ bądź $B = \zero$ wystarczy skorzystać z poprzednich twierdzeń. Natomiast, w przypadku gdy $A = \zero$ lub $B = \zero$, zauważmy, że 
        $$\begin{bmatrix}
            \id & - \id \\ \zero & \id \\
        \end{bmatrix}
        \cdot \begin{bmatrix}
            \id & \zero \\ \id & \id \\
        \end{bmatrix}
        \cdot \begin{bmatrix}
            \id & - \id \\ \zero & \id \\
        \end{bmatrix} 
        \cdot \begin{bmatrix}
            A & B \\ C & D \\ 
        \end{bmatrix} = 
        \begin{bmatrix}
            -C & -D \\ A & B \\
        \end{bmatrix}
        $$
    \end{proof}    
\end{frame}

\begin{frame}{Macierze  blokowe $2 \times 2 $}
    \begin{block}{Twierdzenie}
        Niech będzie dana kwadratowa macierz blokowa $M = \begin{bmatrix}
            A & B \\ C & D \\  
        \end{bmatrix}$
        o wszystkich blokach tego samego rozmiaru. Jeżeli $D$ jest odwracalna, zachodzi 
        $$\det M = \det(AD - BD^{-1}CD)$$
    \end{block}    
    \pause 
    \begin{proof}
        Zauważmy, że 
        $$ \begin{bmatrix}
            A & B \\ C & D \\
        \end{bmatrix}
        \cdot \begin{bmatrix}
            \id & \zero \\ - D^{-1} C & \id  \\
        \end{bmatrix}
        = \begin{bmatrix}
            A - B D^{-1}C & B \\ \zero & D \\  
        \end{bmatrix}
        $$
    \end{proof}
\end{frame}

\begin{frame}{Macierze blokowe $2 \times 2$}
            \begin{alertblock}{Wniosek}
                Jeżeli dodatkowo $CD = DC$ to zachodzi $\det M = \det(AD - BC)$
            \end{alertblock}
\end{frame}

\section{Ogólne Twierdzenie  Laplace'a}

\begin{frame}{Dopełnienie minora macierzy}

    Niech będzie dana macierz kwadratowa $A = \begin{bmatrix}
        a_{11} & \cdots  & a_{1n} \\ 
        \vdots & \ddots & \vdots \\ 
        a_{n1} & \cdots & a_{nn} \\
    \end{bmatrix}$ 
    Niech $M$ będzie minorem wymiaru $p \leq n$.
    Niech $$ j_1 < j_2 \ldots < j_p $$ będą wskaźnikami wierszy, a 
    $$ k_1 < k_2 \ldots < k_p$$ będą wskaźnikami kolumn, z których powstał minor $M$.
    Niech $q = n-p$ oraz $M_1$ będzie minorem wymiaru $q$ powstałym przez wykreślenie wierszy i kolumn należących do minora $M$.
\end{frame}

\begin{frame}{Dopełnienie minora macierzy}
    \begin{block}{Definicja}
        \alert{Dopełnieniem algebraicznym minora $M$} względem macierzy $A$ nazywamy liczbę 
        $$ M^* = (-1)^{\Sigma j + \Sigma k} \cdot M_1.$$
    gdzie $\Sigma j = j_1 + \cdots j_p $ i $\Sigma k = k_1 + \cdots + k_p$.
    \end{block}    
\end{frame}

\begin{frame}{Wieloliniowość wyznacznika}
    \begin{alertblock}{Fakt}
        Niech $A$ będzie macierzą zdefiniowaną jak na poprzednich slajdach. 
        Niech $A_i = \begin{bmatrix}
            a_{i1} & a_{i2} & \ldots & a_{in}
        \end{bmatrix}$ oznacza $i-$ty wiersz macierzy i niech zachodzi 
        $$A_{i} = \begin{bmatrix}
            v_1 & v_2 &\ldots & v_n 
        \end{bmatrix} + \alpha \cdot \begin{bmatrix}
            u_1 & u_2 & \ldots & u_n 
        \end{bmatrix}.$$
        Zachodzi wtedy 
        $$\det \begin{bmatrix}
            a_{11} & \cdots  & a_{1n} \\ 
            \vdots & \ddots & \vdots \\
            a_{i1} & \ldots & a_{in} \\  
            \vdots & \ddots & \vdots \\ 
            a_{n1} & \cdots & a_{nn} \\
        \end{bmatrix}  = 
        \det \begin{bmatrix}
            a_{11} & \cdots  & a_{1n} \\ 
            \vdots & \ddots & \vdots \\
            v_{1} & \ldots & v_{n} \\  
            \vdots & \ddots & \vdots \\ 
            a_{n1} & \cdots & a_{nn} \\
        \end{bmatrix} + \alpha \cdot 
        \det \begin{bmatrix}
            a_{11} & \cdots  & a_{1n} \\ 
            \vdots & \ddots & \vdots \\
            u_{1} & \ldots & u_{n} \\  
            \vdots & \ddots & \vdots \\ 
            a_{n1} & \cdots & a_{nn} \\
        \end{bmatrix}
        $$
    \end{alertblock}
\end{frame}

\begin{frame}{Twierdzenie Laplace'a}
    Zdefinujmy macierz $A$ jak na poprzednich slajdach.
    Rozbijmy macierz $A$ na dwie macierze prostokątne $A'$ oraz $A''$, z których 
    pierwsza składa się z $p$ pierwszych kolumn, a druga z $q$ ostatnich kolumn.
    \begin{block}{Lemat}
        Jeśli w macierzy $A'$ ilość wierszy zawierająca same zera jest większa od $q$ to $\det A = 0$.
    \end{block}
\end{frame}

\begin{frame}{Twierdzenie}
    \begin{block}{Ogólne Twierdzenie Laplace'a}
        Niech $M$ przebiega wszystkie minory jakie można utworzyć
        z dowolnie wybranych i ustalonych kolumn (wierszy) macierzy $A$, natomiast $M^*$ niech będzie algebraicznym dopełnieniem minora $M$.
        Suma wszystkich iloczynów $M M^*$ jest równa $\det A$.
    \end{block}
\end{frame}

\begin{frame}{Źródła}
    \begin{enumerate}
        \item \href{https://www.mdpi.com/2227-7390/11/17/3784}{Inverses and Determinants of $n \times n$ matrices, Müge Saadetoğlu, Şakir Mehmet Dinsev, 2023, \textit{Mathematics}}
        \item \href{https://gubner.ece.wisc.edu/notes/BlockMatrixFormulas.pdf}{Block Matrix Formulas , John A. Gubner, 2024}
        \item Elementy Algebry Wyższej, A. Mostowski, M. Stark, \textit{Wydawnictwo Naukowe PWN }
    \end{enumerate}
\end{frame}
\end{document}