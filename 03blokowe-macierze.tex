\documentclass{beamer}

\usepackage{polski}
\usepackage{pgf,tikz}
\usepackage{tgheros}       

\usetheme{BIT}

\usepackage[utf8]{inputenc}
\usepackage{amssymb,amsmath,amsthm}

\usetikzlibrary{arrows}
\usetikzlibrary{shapes,decorations}

\title{Algebra Komputerowa}
\subtitle{Macierze Blokowe}
\author{Filip Zieli\'nski}
\date{2025}
 
\begin{document}

\begin{frame}
\titlepage
\end{frame}
 
\begin{frame}{Spis Treści}
    \tableofcontents
\end{frame}

\section{Wstęp} 

\begin{frame}{Macierze Blokowe}
    W poniższych rozważaniach skupimy się na macierzach blokowych $2 \times 2$. Wiele wyników da się jednak uogólnić.
    \begin{block}{Definicja}
        Rozważmy macierze $A \in \mathcal{M}_{n_1 \times m_1}, B \in \mathcal{M}_{n_1 \times m_2}, C \in \mathcal{M}_{n_2 \times m_1}, D \in \mathcal{M}_{n_2 \times m_2}$ nad ustalonym ciałem $K$.
        Wtedy, \alert{Macierzą Blokową} (klatkową) $X \in \mathcal{M}_{n_1 + n_2 \times m_1 + m_2}$ złożoną z  $A,B,C,D$ definiujemy jako $X = (x_{ij})$, gdzie 
        $$x_{ij} = \begin{cases}
            a_{ij} & i \leq n_1 , \quad  j \leq m_1 \\ 
            b_{ij} &  i \leq n_1 , \quad m_1 < j \leq m_1 + m_2 \\
            c_{ij} & n_1 < i \leq n_1 + n_2, \quad j \leq m_1 \\
            d_{ij} & n_1 < i \leq n_1 + n_2 \quad  m_1 < j \leq m_1 + m_2 \\
        \end{cases}$$
    \end{block}    
\end{frame}

\begin{frame}{Macierze Blokowe}
    Takie macierze zapisujemy jako
    $$ X = \begin{bmatrix}
        A & B \\
        C & D
    \end{bmatrix}$$
    Zauważmy, że macierze można dzielić na bloki na wiele sposobów, 
    $$X = \left[ \begin{array}{ccc|c}
        1 & 2 & 3 & 4 \\ \hline 
        5 & 6 & 7 & 8 \\ 
        9 & 10 & 11 & 12 \\ 
        13 & 14 & 15 & 16 \\
    \end{array}\right] =  \left[ \begin{array}{cc|cc}
        1 & 2 & 3 & 4 \\ 
        5 & 6 & 7 & 8 \\ \hline 
        9 & 10 & 11 & 12 \\ 
        13 & 14 & 15 & 16 \\
    \end{array}\right] $$
    ale zawsze zachodzi $colA = colC, \quad colB = colD,\quad  rowA = rowB, \quad rowC = rowD$
\end{frame}

\section{Podstawowe Operacje} 

\begin{frame}
    \begin{block}{Obserwacja}
        Niech $X,Y$ będą macierzami tych samych wymiarów. Jeżeli $X,Y$ podzielimy na bloki 
        odpowiednio \textit{tych samych wymiarów}, to macierze blokowe można dodawać
        $$X = \begin{bmatrix}
            A_1 & B_1 \\ C_1 & D_1 \\
        \end{bmatrix},
        Y = \begin{bmatrix}
            A_2& B_2 \\ C_2 & D_2 \\
        \end{bmatrix}, \quad 
        X + Y = \begin{bmatrix}
            A_1 + A_2 & B_1 + B_2 \\ C_1 + C_2 & D_1 + D_2 \\ 
        \end{bmatrix}$$
    \end{block}
    Równie naturalnie, zdefiniowane jest odejmowanie macierzy blokowych jak i mnożenie macierzy przez skalar z ciała.
\end{frame}

\section{Mnożenie Macierzy}

\begin{frame}{Mnożenie Macierzy Blokowych}
    \begin{block}{Twierdzenie}
        Niech, $X \in \mathcal{M}_{n_1 + n_2 \times m_1 + m_2},Y \in \mathcal{M}_{m_1 + m_2 \times p_1 + p_2}$ będą macierzami nad tym samym ciałem $K$ podzielonymi na bloki w następujący sposób
        $$X =\begin{bmatrix}
            A_1 & B_1 \\ C_1 & D_1 \\
        \end{bmatrix}, \quad Y = \begin{bmatrix}
            A_2 & B_2 \\ C_2 & D_2 \\ 
        \end{bmatrix},$$  gdzie $col A_1 = col C_1 = row A_2 = row B_2$ oraz $col B_1 = col D_1 = row C_2 = row D_2 .$
        \newline 
        Wtedy 
        $$ M = XY = \begin{bmatrix}
            A_1 A_2 +  B_1 C_2 & A_1 B_2 + B_1 D_2 \\
            C_1 A_2 + D_1 C_2 & C_1 B_2 + D_1 D_2 \\
        \end{bmatrix}$$
    \end{block}
\end{frame}

\end{document}