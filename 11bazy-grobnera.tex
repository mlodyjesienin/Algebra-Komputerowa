\documentclass{beamer}

\usepackage{polski}
\usepackage{pgf,tikz, tikz-cd}
\usepackage{tgheros}       
\usepackage{bm}

\usetheme{BIT}

\usepackage[utf8]{inputenc}
\usepackage{amssymb,amsmath,amsthm}
\usepackage[backend=bibtex,style=numeric]{biblatex}
\addbibresource{literature.bib}

\usetikzlibrary{arrows}
\usetikzlibrary{shapes,decorations}

\newcommand{\zero}{\mathbf{0}}
\newcommand{\one}{\mathbf{1}}
\newcommand{\ord}{\textrm{ord}}
\newcommand{\II}{\mathnormal{I}}
\newcommand{\JJ}{\mathnormal{J}}
\newcommand{\NWD}{\rm{NWD}}
\newcommand{\NWW}{\rm{NWW}}
\newcommand{\supp}{\textrm{supp}}
\newcommand{\lc}{\textrm{lc}}
\newcommand{\lt}{\textrm{lt}}
\newcommand{\lm}{\textrm{lm}}

\let\phi\varphi
\renewcommand{\epsilon}{\bm{\varepsilon}}

\title{Algebra Komputerowa}
\subtitle{Bazy Gr{\"o}bnera \cite{ComputerAlgebra,LCM,Computative1,Dumnicki}}
\author{Filip Zieli\'nski}
\date{2025}
 
\begin{document}
\begin{frame}
    \titlepage
\end{frame}
 
\begin{frame}{Spis Treści}
    \tableofcontents
\end{frame}

\section{Porządki jednomianowe}
\begin{frame}{Relacje porządkujące}
    \begin{block}{Definicja}
        Porządkiem \only<1>{\alert{częściowym}}\only<2>{\alert{totalnym}} nazywamy relacje $\preceq$ określoną na zbiorze $A$ spełniającą warunki
        \begin{itemize}
            \item \alert{zwrotność} -- $\forall a \in A \quad a \preceq a,$
            \item \alert{antysymetryczność} -- $\forall a,b \in A \quad (a \preceq b \land b \preceq a) \Rightarrow  a = b,$
            \item \alert{przechodniość} -- $\forall a,b,c \in A \quad  (a \preceq b \land b \preceq c) \Rightarrow a \preceq b,$
            \only<2>{\item \alert{spójność} -- $\forall a,b \in A \quad a \preceq b \lor b \preceq a$.}
        \end{itemize}
    \end{block}
\end{frame}

\begin{frame}{Oznaczenia i Definicja}
    \begin{alertblock}{Notacja}
        \begin{itemize}
            \item $K[\mathbb{X}] = K [x_1, \ldots, x_n],$
            \item $\alpha = (\alpha_1, \ldots, \alpha_n),$
            \item $c_{\alpha} \mathbb{X}^{\alpha} = c_{\alpha} x_1^{\alpha_1}x_2^{\alpha_2}\cdots x_n^{\alpha_n},$
            \item $\mathbb{M}^n = \{\mathbb{X}^{\alpha} \mid \alpha \in \mathbb{N}^n_{0}\}.$
        \end{itemize}
    \end{alertblock}
    \pause 
    \begin{block}{Definicja}
        Porządek totalny $\preceq$ określony na zbiorze $\mathbb{M}^n$ nazywamy \alert{jednomianowym}, jeżeli spełnione są następujące warunki 
        \begin{itemize}
            \item $1 \preceq \mathbb{X}^{\alpha}$ dla każdego $\alpha \in \mathbb{N}^n_{0}.$
            \item każdy niepusty zbiór $S \subset \mathbb{M}^n$ posiada element najmniejszy.
            \item $\mathbb{X}^{\alpha} \preceq \mathbb{X}^{\beta} \Rightarrow \mathbb{X}^{\alpha}\mathbb{X}^{\gamma} \preceq \mathbb{X}^{\beta}\mathbb{X}^{\gamma}.$ dla każdego $\gamma \in \mathbb{N}^{n}_{0}.$
        \end{itemize}
    \end{block}
\end{frame}

\begin{frame}{Przykłady}
    \begin{exampleblock}{Przykład porządku jednomianowego}
            \textit{Porządek leksykograficzny (lex)} 
            $$ \mathbb{X}^{\alpha} \preceq \mathbb{X}^{\beta} \Leftrightarrow \alpha_1 = \beta_1, \ldots \alpha_s = \beta_s, \alpha_{s+1} < \beta_{s+1}$$
    \end{exampleblock}
    \pause 
    Inne znane i wykorzystywane porządki to np.
    \begin{itemize}
        \item porządek stopniowo leksykograficzny,
        \item porządek odwrotny do leksykograficznego,
        \item stopniowy porządek odwrotny do leksykograficznego.
    \end{itemize}
\end{frame}

\section{Redukcje wielomianowe}
\begin{frame}{Notacja}
    \begin{block}{Definicja}
        Niech będzie dany porządek jednomianowy $\preceq$ oraz wielomian wielu zmiennych 
        $$f = c_{\alpha_1}\mathbb{X}^{\alpha_1} + \ldots + c_{\alpha_m}\mathbb{X}^{\alpha_m},$$ gdzie
        $\mathbb{X}^{\alpha_1} \succeq \ldots \succeq \mathbb{X}^{\alpha_m}.$
        Wtedy 
        \begin{itemize}
            \item \alert{Nośnikiem} wielomianu $f$ nazywamy zbiór wszystkich jego jednomianów 
            $$ \supp \ f := \{\mathbb{X}^{\alpha_1}, \ldots, \mathbb{X}^{\alpha_m}\}$$
            \item \alert{Jednomianem wiodącym} $f$ nazywamy
            $\lm_{\preceq}(f)
            = \mathbb{X}^{\alpha_1}.$
            \item \alert{Współczynnikiem wiodącym} $f$ nazywamy $\lc_{\preceq}(f) = c_{\alpha_1}.$
            \item \alert{Wyrazem wiodącym} $f$ nazywamy $\lt_{\preceq}(f) = \lc_{\preceq}(f) \cdot \lm_{\preceq}(f).$
        \end{itemize}
    \end{block}
\end{frame}

\begin{frame}{Redukcja Wielomianowa}
    \begin{block}{Definicja}
        \begin{itemize}
        \item Mówimy, że wielomian $f$ redukuję się jednym kroku do wielomianu $h$,  modulo wielomian $g$, co oznaczamy $f \xrightarrow{g} h $, jeżeli istnieje $\mathbb{X}^{\alpha} \in \supp \ f$, taki, że 
        $$ \lm(g) \mid \mathbb{X}^{\alpha}  \text{  oraz  } h = f - \frac{c_{\alpha} \mathbb{X}^{\alpha} }{\lt (g)} g.$$

        \item Wielomian $f$ redukuje się do wielomianu $h$ modulo $G = \{g_1, \ldots , g_s\}$, co oznaczamy $f \xrightarrow{G} h$, jeżeli 
        $$ f = f_0 \xrightarrow{g_{i1}} f_1 \xrightarrow{g_{i2}}  \ldots  \xrightarrow{g_{im}} f_m =  h.$$
        
        \item Jeżeli wielomianu $h$ nie można bardziej zredukować modulo $G$, to mówimy, że $h$ jest \alert{resztą} wielomianu $f$ modulo $G$.
        \end{itemize}
    \end{block}
\end{frame}

\begin{frame}{Jednoznaczność redszty modulo}
    \begin{alertblock}{Uwaga}
        Reszta wielomianu $f$ modulo $G = \{g_1, \ldots, g_s\}$ dla wielomianów wielu zmiennych \textbf{nie} jest wyznaczona jednoznacznie.
    \end{alertblock}
\end{frame}

\begin{frame}{Algorytm Wyznaczania reszt modulo $G$}
    \begin{alertblock}{Obserwacja}
        W tym wypadku najprostsze podejście polega na brutalnym sprawdzaniu, czy da się wykonać krok redukcji, jeżeli tak to go wykonujemy, jeżeli nie, to znaleźliśmy już resztę.
    \end{alertblock}
\end{frame}

\section{Bazy Gr{\"o}bnera}
\begin{frame}{Ideały jednomianowe}
    \begin{block}{Twierdzenie}
        Niech $\II \triangleleft K[\mathbb{X}]$ będzie ideałem oraz zbiór $G$ jego skończonym zbiorem generatorów. Wszystko rozważamy w ustalonym porządkun jednomianowym $\preceq$. Następujące warunki są równoważne 
        \begin{enumerate}
            \item $\langle \lt(g) \mid g \in g \rangle = \langle \lt (f) \mid f \in \II \rangle =: \lt (\II).$
            \item $f \in \II , f \neq 0  \Rightarrow \lt (g) \mid \lt(f) $ dla pewnego $g \in G$. 
            \item Reszty modulo $G$ są jednoznaczne. 
            \item $f \in \II \Leftrightarrow f \xrightarrow{G} 0.$
        \end{enumerate}
    \end{block}
\end{frame}

\begin{frame}{Bazy Gr\"obnera}
    \begin{block}{Definicja}
        Zbiór $G$ spełniający warunki poprzedniego twierdzenia nazywamy \alert{bazą Gr\"obnera} ideału $\II$.
    \end{block}
    \pause 
    \begin{alertblock}{Uwaga}
        Każdy ideał $\II \triangleleft K[\mathbb{X}]$ posiada bazę Gr\"obnera.
    \end{alertblock}
\end{frame}

\begin{frame}{S-wielomiany}
    \begin{block}{Definicja}
        Niech $f,g \in K[\mathbb{X}]$ będą wielomianami. $S-$wielomianem $f,g$ nazywamy wielomian zadany wzorem 
        $$ S(f,g) = \frac{\textrm{lcm} (\lm(f),\lm(g))}{\lt(f)}f -\frac{\textrm {lcm}(\lm(f),\lm(g))}{\lt(g)}g $$
    \end{block}
\end{frame}

\begin{frame}{Kryterium Buchbergera}
    \begin{block}{Twierdzenie (Buchberger)}
        Niech $\II \triangleleft K[\mathbb{X}]$ będzie ideałem w pierścieniu wielomianów wielu zmiennych oraz niech $G = \{g_1 , \ldots , g_n\}$ będzie skończonymz zbiorem generatorów $\II$. 

        Następujące warunki są równoważne:
        \begin{enumerate}
            \item $G$ jest bazą Gr\"obnera. 
            \item $S(g_i,g_j) \xrightarrow{G} 0,$ dla każdego $1 \leq i < j \leq n$. 
        \end{enumerate}
    \end{block}
\end{frame}

\begin{frame}{Algorytm \textit{is\_Gr\"obner\_basis}}
    \textbf{Wejście:} 
    \begin{itemize} 
        \item Skończony zbiór $G \subset K[\mathbb{X}],$ \item porządek jednomianowy $\preceq$. 
    \end{itemize}

    \textbf{Wyjście:} 
    \begin{itemize}
        \item Wartość logiczna prawda - $G$ jest bazą Gr\"obnera ideału $\langle G \rangle,$ 
        \item wartość logiczna fałsz wpp. wraz z niezerową resztą $S-$wielomianu dwóch wielomianów z $G$.
    \end{itemize}

    \textbf{Kroki:}
    \begin{enumerate}
        \item Dla każdej pary wielomianów $f,g \in G$, $f \neq g$:
        \begin{itemize}
            \item Utwórz ich $S-$wielomian, 
            \item wyznacz resz†ę $r$ wielomianu  $S(f,g)$ modulo $G$, 
            \item jeżeli jest niezerowa zwróć (\textit{fałsz},$r$).
        \end{itemize}
        \item Zwróć \textit{prawda}.
    \end{enumerate}
\end{frame}

\begin{frame}{Algorytm Buchbergera}
        \textbf{Wejście:} 
    \begin{itemize} 
        \item Skończony zbiór generatorów $F$ ideału $\II$.
    \end{itemize}

    \textbf{Wyjście:} 
    \begin{itemize}
        \item Baza Gr\"obnera $G$ ideału $\II$.
    \end{itemize}

    \textbf{Kroki:}
    \begin{enumerate}
        \item Zainicjalizuj $G := F$, 
        \item dopóki \textit{is\_Gr\"obner\_basis($G$)} to fałsz wykonuj: 
        \begin{itemize}
            \item $G := G \cup \{r \}$, gdzie $r$ jest niezerową resztą zwróconą przez \textit{is\_Gr\"obner\_basis($G$)}.
        \end{itemize}
        \item Zwróć $G$.
    \end{enumerate}
\end{frame}

\begin{frame}{Minimalne bazy Gr\"obnera}
    \begin{block}{Twierdzenie}
        Niech $G$ będzie bazą Gr\"obnera idału $\II \triangleleft K[\mathbb{X}]$ oraz $f,g \in G$, $f \neq g$ będą wielomianami z bazy. 

        Jeżeli $\lm(g) \mid \lm(f)$, to $G \setminus \{f\}$ też jest bazą Gr\"obnera ideału $\II$.
    \end{block}
    \pause 
    \begin{block}{Definicja}
        Bazę Gr\"obnera ideału $\II$ nazywamy minimalną jeżeli zachodzi: 
        $$ \forall f,g \in G, f \neq g \quad \lm(g) \nmid \lm(f)$$
    \end{block}
\end{frame}

\begin{frame}{Algorytm}
    \begin{block}{Obserwacja}
        Brutalny algorytm minimalizujący bazę Gr\"obnera $G$ jest oczywisty.
    \end{block}
\end{frame}

\begin{frame}{Zredukowana baza Gr\"obnera}
    \begin{block}{Definicja}
    Bazę Gr\"obnera $G$ nazywamy zredukowaną jeżeli 
    \begin{itemize}
        \item $\forall f \in G  \lc(f) = 1$
        \item dla wszystkich $f,g \in G$ oraz dla wszystkich $\mathbb{X}^{\alpha} \in \supp \ f$ zachodzi 
        $$ \lm(g) \mid \mathbb{X}^{\alpha}.$$
    \end{itemize}
    \end{block}
    \pause 
    \begin{alertblock}{Obserwacja}
        Redukowanie minimalnych baz Gr\"obnera sprowadza się do przeprowadzania redukcji modulo $G$ wszystkich wielomianów z $G$. 
    \end{alertblock}
    \pause 
    \begin{block}{Twierdzenie}
        Każdy ideał $\II \triangleleft K[\mathbb{X}]$ posiada dokładnie \alert{dokładnie jedną} minimalną zredukowaną bazę Gr\"obnera.
    \end{block}
\end{frame}

\section{Eliminacja Zmiennych}
\begin{frame}{Twierdzenie o Eliminacji}
    \begin{block}{Twierdzenie}
        Rozważmy pierścień wielomianów wielu zmiennych $K[x_1, \ldots, x_m, y_1, \ldots, y_r] = K[\mathbb{X}, \mathbb{Y}].$

        Niech $\preceq$ będzie porządkiem jednomianowym na $K[\mathbb{X},\mathbb{Y}]$ takim, że
        $$ \forall \alpha \in \mathbb{N}_{0}^{m}, \beta \in \mathbb{N}_{0}^{r} \quad \mathbb{X}^{\alpha} \succeq \mathbb{Y}^{\beta} $$
        oraz $G$ będzie bazą Gr\"obnera (względem porządku $\preceq$) ideału $\II \triangleleft K[\mathbb{X}, \mathbb{Y}]$. 
        Wtedy 
        $$ G \cap K[\mathbb{Y}]$$
        jest bazą Gr\"obnera ideału $\II \cap K[\mathbb{Y}] \triangleleft K[\mathbb{Y}]$, względem porządku jednomianowego $\left.\preceq \right|_{K[\mathbb{Y}]}.$
    \end{block}    
\end{frame}

\begin{frame}{Źródła}
    Prezentacja jest mocno oparta o wykład autorstwa \textit{Przemysława Koprowskiego}, który można obejrzeć pod tym 
    \href{https://www.youtube.com/watch?v=vdmyrbNqRlY&t=6452s}{linkiem}
    \printbibliography
\end{frame}

\begin{frame}
    \centering 
    \LARGE Pytania, wątpliwości, uwagi ? 
\end{frame}

\end{document}