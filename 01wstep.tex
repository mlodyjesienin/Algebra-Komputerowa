\documentclass{beamer}

\usepackage{polski}
\usepackage{pgf,tikz}
\usepackage{tgheros}       

\usepackage{qrcode}


\usetheme{BIT}

\usepackage[utf8]{inputenc}
\usepackage[backend=bibtex,style=numeric]{biblatex}
\addbibresource{literature.bib}
\usepackage{amssymb,amsmath,amsthm}

\usetikzlibrary{arrows}
\usetikzlibrary{shapes,decorations}

\title{Algebra Komputerowa}
\subtitle{Informacje Wstępne}
\author{Filip Zieli\'nski}
\date{2025}
 
\begin{document}

\begin{frame}
    \titlepage
\end{frame}
 
\begin{frame}{Podstawowe Informacje}
    \begin{itemize}
        \item Zajęcia co tydzień w środę 18:30 sala 3.27d, Budynek D17 
        \item Repozytorium z prezentacjami: 
                \href{https://github.com/mlodyjesienin/Algebra-komputerowa}{\textit{github.com/mlodyjesienin/Algebra-komputerowa}}
        \item mail: \textit{fzielinski@student.agh.edu.pl}
    \end{itemize}
    \begin{center}
        \begin{tabular}{p{.3\textwidth} p{.3\textwidth}}
        \qrcode{https://discord.gg/wf4PdGBen9}  & \qrcode{https://github.com/mlodyjesienin/Algebra-komputerowa} \\[3em]
        Discord  &   Github
        \end{tabular}
    \end{center}
\end{frame}

\begin{frame}{Lista Tematów}
    \begin{enumerate}
        \item Powtórzenie Podstaw Algebry 
        \item Macierze Blokowe
        \item Teoria Grup, Podgrupy Normalne, Grupy Cykliczne \cite{Gleichgewicht}
        \item Teoria Pierścieni, Ideały, Struktury Ilorazowe \cite{Gleichgewicht}
        \item Rozszerzony Algorytm Euklidesa \cite{ComputerAlgebra}
        \item Algorytmy Modularne, Chińskie Twierdzenie o Resztach \cite{LCM} 
        \item Rozkład Bezkwadratowy Wielomianu \cite{LCM}, \cite{ComputerAlgebra}
        \item Bazy Gr{\"o}bnera \cite{Dumnicki}, \cite{Computative1}, \cite{ComputerAlgebra}, \cite{LCM}
    \end{enumerate}
        
\end{frame}

\begin{frame}{Potencjalne Tematy}
    \begin{enumerate}
        \item Rozkład Liczby na Czynniki Pierwsze
        \item Algebra Przemienna, Moduły, k-Algebry 
        \item Wstęp do Geometrii Algebraicznej 
    \end{enumerate}
\end{frame}

\begin{frame}{Literatura}
    \nocite{*}
    \printbibliography
\end{frame}

\end{document}