\documentclass{beamer}

\usepackage{polski}
\usepackage{pgf,tikz, tikz-cd}
\usepackage{tgheros}       
\usepackage{bm}

\usetheme{BIT}

\usepackage[utf8]{inputenc}
\usepackage{amssymb,amsmath,amsthm}
\usepackage[backend=bibtex,style=numeric]{biblatex}
\addbibresource{literature.bib}

\usetikzlibrary{arrows}
\usetikzlibrary{shapes,decorations}

\newcommand{\zero}{\mathbf{0}}
\newcommand{\one}{\mathbf{1}}
\newcommand{\ord}{\textrm{ord}}
\newcommand{\II}{\mathnormal{I}}
\newcommand{\JJ}{\mathnormal{J}}
\newcommand{\NWD}{\rm{NWD}}
\newcommand{\NWW}{\rm{NWW}}
\newcommand{\rad}{\textrm{rad}}

\let\phi\varphi
\renewcommand{\epsilon}{\bm{\varepsilon}}

\title{Algebra Komputerowa}
\subtitle{Rozkład Bezkwadratowy Wielomianu \cite{ComputerAlgebra,LCM}}
\author{Filip Zieli\'nski}
\date{2025}
 
\begin{document}
\begin{frame}
    \titlepage
\end{frame}
 
\begin{frame}{Spis Treści}
    \tableofcontents
\end{frame}

\section{Elementy bezkwadratowe}
\begin{frame}{Założenia}
    \begin{itemize}
        \item Przez całą prezentacje zakładamy, że $K$ jest ciałem \textbf{charakterystyki zero}, bądź ciałem \textbf{skończonym}.
        \item Uwaga! Przedstawione twierdzenia nie muszą zachodzić dla ciał nieskończonych o skończonej charakterystyce.
        \item Mówiąc o pierścieniu, musimy mówić o pierścieniu z jednoznacznością rozkładu.
        \item Zazwyczaj myślimy o pierścieniu wielomianów jednej zmiennej $K[x]$.
    \end{itemize}
\end{frame}

\begin{frame}{Bezkwadratowość}
    \begin{block}{Definicja}
        Niech $R$ będzie pierścieniem oraz $a \in R$ elementem tego pierścienia. $a$ nazywamy \textit{bezkwadratowym}, jeżeli nie jest podzielny przez kwadrat żadnego elementu nieodwracalnego.
    \end{block}
\end{frame}

\begin{frame}{Wielomiany Bezkwadratowe}
    \begin{alertblock}{Obserwacja}
        Niech $f \in K[x]$ będzie wielomianem o rozkładzie 
        $$f = \epsilon \cdot p_1^{\alpha_1} \cdots p_k^{\alpha_m} $$ 
        gdzie $\epsilon$ jest elementem odwracalnym, a $p_1,\ldots, p_m$ są różnymi wielomianami unormowanymi nierozkładalnymi. 

        Następujące warunki są równoważne
        \begin{enumerate}
            \item $f$ jest bezkwadratowy,
            \item $\alpha_1 = \ldots = \alpha_m = 1$,
            \item rozważając $f$ w swoim ciele rozkładu, $f$ posiada jedynie pierwiastki jednokrotne.
        \end{enumerate}
    \end{alertblock}
\end{frame}

\begin{frame}{Pierwiastki wielokrotne wielomianu}
    \begin{block}{Lemat}
        Niech $f \in K[x]$ będzie wielomianem, a $z$ pierwiastkiem $f$ o krotności $k$.
        Jeżeli $k > 1$ to $f'(z) = 0$. 
    \end{block}
    \pause 
    \begin{block}{Twierdzenie}
        Niech $f \in K[x]$ będzie wielomianem jednej zmiennej o współczynnikach z ciała $K$. Zachodzi 
        $$f \text{ jest bezkwadratowy} \Leftrightarrow \NWD(f,f') = 1$$
    \end{block}
\end{frame}

\section{Rozkład bezkwadratowy wielomianu}
\begin{frame}{Definicja}
    Niech $f \in K[x]$ będzie wielomianem o rozkładzie 
    $$f = \epsilon \cdot p_1^{\alpha_1} \cdots p_k^{\alpha_m}.$$
    Oznaczmy $k = \max \{\alpha_1 , \ldots \alpha,_m\}$.
    Pogrupujmy czynniki względem ich krotności, tzn 
    $$ g_i = \prod_{j \leq m, \ \alpha_j = i}p_j \quad i \leq k $$
    \begin{block}{Definicja}
        \textit{Rozkładem bezkwadratowym} wielomianu $f$ nazywamy zapisanie go w postaci
        $$f = \epsilon \cdot g_1^{1} \cdot g_2^2 \cdots g_k^k$$
    \end{block}
\end{frame}

\begin{frame}{Radykał Wielomianu}
    \begin{block}{Definicja}
        Niech $f \in K[x]$ będzie wielomianem o rozkładzie postaci
        $$f =\epsilon \cdot p_1^{\alpha_1} \cdots p_k^{\alpha_m}.$$
        \textit{Radykałem wielomianu} $f$ nazywamy wielomian 
        $$ \rad(f) = p_1 \cdots p_m.$$
    \end{block}
    \pause
    \begin{block}{Twierdzenie}
        Niech $f$ będzie wielomianem  o współczynnikach z \textbf{ciała charakterystyki zero}.
        Zachodzi 
        $$\rad (f) = g_1 \cdots g_k = \frac{f}{\NWD(f,f')}$$
    \end{block}
\end{frame}

\section{Algorytm Tobeya - Horowitza}
\begin{frame}{Algorytm rozkładu bezkwadratoego}
    \textbf{Założenia:} $K$ -- ciało charakterystyki zero. 

    \textbf{Wejście:} $f \in K[x]$

    \textbf{Wyjście:} $g_1,\ldots, g_k$ takie, że $f = \epsilon \prod_{i=1}^{k}g_i^i$
    
    \textbf{Kroki:}
    \begin{enumerate}
        \item Przypisz $a_0 := f$, $a_1 := \NWD(a_0, a_0')$, $b_1 = \frac{a_0}{a_1}$, $i:=1$. 
        \item Dopóki $b_i \neq 1$, wykonuj:
        \begin{itemize}
            \item Wylicz $a_{i+1} = \NWD(a_i, a_i')$.
            \item Wylicz $b_{i+1} = \frac{a_{i}}{a_{i+1}}$.
            \item Wylicz $g_i = \frac{b_i}{b_{i+1}}$.
        \end{itemize}
        \item Zwróć $g_1, \ldots, g_k$. 
    \end{enumerate}
\end{frame}

\begin{frame}{Zaawansowane algorytmy}
    Algorytm Tobeya - Horowitza jest \textbf{najstarszym} znanym algorytmem rozkładania bezkwadratowego wielomianu.
    Został wymyślony w latach 1967/1969. 
    \pause 
    \begin{itemize}
        \item Drobna modyfikacja tego algorytmu, sprawia, że działa on na wielomianach nad ciałami skończonymi. 
        \pause \item Istnieje szereg szybszych algorytmów \begin{enumerate}
            \item algorytm Mussera (1971),
            \item algorytm Yuna (1976),
            \item algorytm Gerharda (2001),
            \item algorytm Guersenzvaiga-Szechtmana (2012/2017).
        \end{enumerate}
    \end{itemize}
\end{frame}
\begin{frame}{Źródła}
    Prezentacja jest mocno oparta o wykład autorstwa \textit{Przemysława Koprowskiego}, który można obejrzeć pod tym 
    \href{https://www.youtube.com/watch?v=x8AMniwrD4s&t=3373s}{linkiem}
    \printbibliography
\end{frame}

\begin{frame}
    \centering 
    \LARGE Pytania, wątpliwości, uwagi ? 
\end{frame}

\end{document}