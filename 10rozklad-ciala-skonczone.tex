\documentclass{beamer}

\usepackage{polski}
\usepackage{pgf,tikz, tikz-cd}
\usepackage{tgheros}       
\usepackage{bm}

\usetheme{BIT}

\usepackage[utf8]{inputenc}
\usepackage{amssymb,amsmath,amsthm}
\usepackage[backend=bibtex,style=numeric]{biblatex}
\addbibresource{literature.bib}

\usetikzlibrary{arrows}
\usetikzlibrary{shapes,decorations}

\newcommand{\zero}{\mathbf{0}}
\newcommand{\one}{\mathbf{1}}
\newcommand{\ord}{\textrm{ord}}
\newcommand{\II}{\mathnormal{I}}
\newcommand{\JJ}{\mathnormal{J}}
\newcommand{\NWD}{\rm{NWD}}
\newcommand{\NWW}{\rm{NWW}}
\newcommand{\rad}{\textrm{rad}}


\let\phi\varphi
\renewcommand{\epsilon}{\bm{\varepsilon}}

\title{Algebra Komputerowa}
\subtitle{Rozkład Wielomianu nad Ciałem Skończonym \cite{ComputerAlgebra,LCM}}
\author{Filip Zieli\'nski}
\date{2025}
 
\begin{document}
\begin{frame}
    \titlepage
\end{frame}
 
\begin{frame}{Spis Treści}
    \tableofcontents
\end{frame}

\section{Wstęp}
\begin{frame}{Oznaczenia}
    Ciało skończone liczności $q$ oznaczmy przez $\mathbb{F}_q$. 
    
    Oczywiście $q = p^m$ dla pewnego $p$ będącego liczbą pierwszą i dla dodatniego całkowitego wykładnika $m$. 

     Zachodzi $char(\mathbb{F}_q)= p$.
\end{frame}

\begin{frame}{Plan}
    Algorytm dzielenia wielomianów nad ciałem skończonym dzielimy na trzy etapy,
    \begin{enumerate}
        \item \textit{Squarefree factorization} (rozkład ze względu na krotność).
        \item \textit{Distinct-degree factorization} (podział ze względu na stopień)
        \item \textit{Equal degree factorization} (ostateczny rozkład)
    \end{enumerate}
    \pause 
    Temat rozkładu bezkwadratowego wielomianu poruszała poprzednia prezentacja.
    Dzięki temu możemy rozważać problem rozkładu wielomianu $g$, który jest bezkwadratowy.
\end{frame}

\section{Distinct-degree factorization}
\begin{frame}{Definicja}
    Rozważmy bezkwadratowy unormowany wielomian $g \in \mathbb{F}_q[x]$ o rozkładzie
    $$g = p_1 \cdots p_m.$$
    Oznaczmy $s = \max \{ \deg p_i \mid i \leq m\}.$
    Pogrupujmy czynniki $g$ względem stopnia, znaczy się 
    $$h_k = \prod_{\deg p_i = k}p_i \quad \text{dla } k \leq s.$$
    \begin{block}{Definicja}
        Rozkładem \textit{distinct-degree} wielomianu $g$ nazywamy wyrażenie 
        $$ g = h_1 \cdots h_s.$$
    \end{block}
\end{frame}

\begin{frame}{Kluczowe twierdzenie}
    \begin{block}{Twierdzenie}
        Ustalmy liczbę całkowitą $d \geq 1$ oraz niech $p_1, \ldots, p_t \in \mathbb{F}_q[X]$ będą 
        \textbf{wszystkimi} unormowanymi nierozkładalnymi wielomianami o współczynnikach z $\mathbb{F}_q$ o takim stopniu, że $\deg(p_i) \mid d$ dla każdego $i$.
        Zachodzi 
        $$p_1 \cdots p_t = x^{q^d} - x.$$
    \end{block}
\end{frame}

\begin{frame}{Algorytm}
 \textbf{Założenia:}  
\( g \in \mathbb{F}_q[x] \) – wielomian unormowany bezkwadratowy.

\textbf{Wejście:}  
Wielomian \( g \in \mathbb{F}_q[x] \).

\textbf{Wyjście:}  
Wielomiany \( h_1, \ldots, h_s \in \mathbb{F}_q[x] \), takie że  
\[
g = h_1 h_2 \cdots h_s,
\]
gdzie każdy \( h_i \) jest iloczynem nierozkładalnych czynników stopnia dokładnie \( i \).

\textbf{Kroki:}
\begin{enumerate}
    \item Przypisz \( f_0 := x \), \( g_1 := g \), \( k := 1 \).
    \item Dopóki \( g_k \) jest wielomianem niebędącym stałą, wykonuj:
    \begin{itemize}
        \item Oblicz \( f_k := f_{k-1}^q \bmod g_k \).
        \item Oblicz \( h_k := \gcd(g_k, f_k - x) \).
        \item Zwiększ \( k := k + 1 \).
        \item Przypisz \( g_k := \frac{g_{k-1}}{h_{k-1}} \).
    \end{itemize}
    \item Zwróć \( h_1, \ldots, h_{k} \).
\end{enumerate}
\end{frame}

\section{Equal degree factorization}
\begin{frame}{Pozostałości}
    Musimy teraz rozłożyć wielomian $f$ stopnia $n$ o współczynnikach z ciała $\mathbb{F}_q$, o rozkładzie 
    $$f = p_1 \cdots p_ m$$ gdzie $\deg (p_i) = d$ dla każdego $p$. 
    \pause 
    Oznaczmy sobie $R = \mathbb{F}_q[x]/\langle f \rangle$ oraz  $K_i = \mathbb{F}_q / \langle p_i \rangle$.
    
    Zauważmy, że $K_i$ jest rozszerzeniem ciała $\mathbb{F}_q$ stopnia $d$ dla każdego $K_i$. 
    Tak więc $K_i \cong K_j$ dla każdego $i,j$.

    Dodatkowo, zauważmy, że $R \cong \mathbb{F}_q[x]_{\leq n-1}.$
\end{frame}    

\begin{frame}{Izomorfizm}
    \begin{alertblock}{Obserwacja}
    Z chińskiego twierdzenia o resztach wynika istnienie izomorfizmu $\Phi$, takiego, że
    $$ \Phi : R \leftarrow K_1 \times \ldots \times K_m.$$        
    \end{alertblock}
\end{frame}

\begin{frame}{Algortym Cantora - Zassenhausa (1981)}
    Rozważmy $g$ względnie pierwsze z $f$. Oznaczmy $\tilde{g} = g \pmod{f}$.
    Mamy $$ \Phi(\tilde{g}) = (\tilde{g_1}, \ldots, \tilde{g}_m)$$
    ustalmy $e = \frac{q^d - 1}{2}$. Zauważmy, że 
    $$ \Phi(\tilde{g}^e) = (\tilde{g_1}^e, \ldots, \tilde{g}_m^e) = (\pm 1, \ldots , \pm 1)$$
    \pause 
    \begin{block}{Twierdzenie}
        Niech $h$ będzie resztą z dzielenia $g^e -1$ przez $f$. Jeżeli jest takie $i \neq j$,
        że $\tilde{g_i}^e =1$ oraz $\tilde{g_j}^e = -1$, to $\NWD(h,f)$ jest nietrywialnym dzielnikiem $f$.  
    \end{block}
\end{frame}

\begin{frame}{Źródła}
    \printbibliography
\end{frame}

\begin{frame}
    \centering 
    \LARGE Pytania, wątpliwości, uwagi ? 
\end{frame}

\end{document}