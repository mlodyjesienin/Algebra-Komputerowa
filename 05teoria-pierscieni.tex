\documentclass{beamer}

\usepackage{polski}
\usepackage{pgf,tikz, tikz-cd}
\usepackage{tgheros}       

\usetheme{BIT}

\usepackage[utf8]{inputenc}
\usepackage{amssymb,amsmath,amsthm}
\usepackage[backend=bibtex,style=numeric]{biblatex}
\addbibresource{literature.bib}

\usetikzlibrary{arrows}
\usetikzlibrary{shapes,decorations}

\newcommand{\zero}{\mathbf{0}}
\newcommand{\one}{\mathbf{1}}
\newcommand{\ord}{\textrm{ord}}
\newcommand{\II}{\mathnormal{I}}
\newcommand{\JJ}{\mathnormal{J}}

\let\phi\varphi

\title{Algebra Komputerowa}
\subtitle{Elementy Teorii Pierścieni \cite{Gleichgewicht,Rusek}}
\author{Filip Zieli\'nski}
\date{2025}
 
\begin{document}
\begin{frame}
    \titlepage
\end{frame}
 
\begin{frame}{Spis Treści}
    \tableofcontents
\end{frame}

\section{Pierścienie}
\begin{frame}{Powtórka}
    \begin{block}{Definicja}
        Zbiór $R$ z dwoma działaniami $+, \cdot$ nazywamy \only<1>{\alert{Pierścieniem},} \only<2>{\alert{Pierścieniem z jedynką},} \only<3,4>{\alert{Pierścieniem przemiennym z jedynką},} jeżeli zachodzą następujące warunki
        \begin{enumerate}
            \item $(R, +)$ jest grupą abelową 
            \item $(R, \cdot)$ jest \only<1>{półgrupą} \only<2>{monoidem} \onslide<3->{monoidem przemiennym}
            \item $\forall x,y,z \in R \quad x \cdot (y + z) = x \cdot y + x \cdot z \land $ \\ $(x + y) \cdot z = x \cdot z + y \cdot z$ \hfill \textit{(rozdzielność mn. wzg. dod.)} 
        \end{enumerate}
    \end{block}
    \onslide<4->{
        \begin{alertblock}{Uwaga}
            Od tego momentu rozważamy jedynie pierścienie przemienne z jedynką. O ile nie jest powiedziane inaczej, gdy piszemy \textit{pierścień} mamy na myśli \textit{pierścień przemienny z jedynką.}
        \end{alertblock}
    }
\end{frame}

\begin{frame}{Przykłady}
    \begin{exampleblock}{Kluczowe pierścienie}
        Z naszej perspektywy najistotniejszymi pierścieniami będą
        \begin{itemize}
            \item $\mathbb{Z}$ - pierścień liczb całkowitych.
            \item $R[x]$ - pierścień wielomianów jednej zmiennej o współczynnikach z pierścienia $R$.
            \item $R[x_1, \ldots , x_n]$ - pierścienie wielomianów wielu zmiennych.
        \end{itemize}
    \end{exampleblock}
\end{frame}

\section{Ideały}
\begin{frame}{Ideały}
    \begin{block}{Definicja}
        Niepusty podzbiór $\II $ pierścienia $R$ nazywa się \alert{ideałem}, jeżeli zachodzi 
        \begin{align}
            & \forall a,b \in \II  \quad a + b \in \II  \\
            & \forall a \in \II \ \forall r \in R \quad r \cdot a \in \II
        \end{align}
    \end{block}
    \pause 
    \begin{alertblock}{Konwencja}
        Fakt, że $\II$ jest ideałem pierścienia $R$ oznaczamy przez $\II \triangleleft R$.
    \end{alertblock}
    \begin{alertblock}{Uwaga}
        Jeżeli potraktujemy $R$ jako grupę z dodawaniem, to $\II$ jest podgrupą normalną grupy $R$.
    \end{alertblock}
\end{frame}

\begin{frame}{Przykłady Ideałów}
    \begin{exampleblock}{Przykład}
        Ideałami są np. 
        \begin{enumerate}
            \item Każdy pierścień $R$ posiada ideał trywialny $\{ \zero \}$ oraz ideał niewłaściwy $R$.
            \item W pierścieniu $\mathbb{Z}$ ideałami są zbiory postaci $n\mathbb{Z} = \{n \cdot k  \mid k \in \mathbb{Z} \}$ - zbiory wielokrotności danej liczby.
            \item W pierścieniu wielomianów $R[x]$ weźmy ustalony wielomian $f$. Zbiór $\II = \{fg \mid g \in R[x]\}$ jest ideałem w $R[x]$.
            \item W pierścieniu wielomianów $R[x_1, \ldots , x_n]$ weźmy ustalone wielomiany $f,g$. Zbiór $\II = \{ a_1f + a_2g \mid a_1,a_2 \in R[x_1, \ldots x_n]\}$ jest ideałem w pierścieniu $R[x_1, \ldots, x_n]$.
        \end{enumerate}
    \end{exampleblock}
\end{frame}

\begin{frame}{Ideały Ciała}
    \begin{block}{Twierdzenie}
        Niech $R$ będzie pierścieniem, a $\II$ ideałem tego pierścienia. Zachodzi 
        $$ \II = R \Leftrightarrow \one \in \II$$
    \end{block}
    \pause 
    \begin{proof}
        ($\Rightarrow$). Oczywiste. \\ 
        ($\Leftarrow$). Jeżeli $\one \in \II$ to z definicji dla dowolnego $r \in R$ zachodzi $r \cdot \one \in \II$, czyli $r \in \II$.
    \end{proof}
    \pause 
\end{frame}

\begin{frame}{Ideały Ciała}
    \begin{block}{Twierdzenie}
        Jeżeli $K$ jest ciałem, to jedynymi jego ideałami są $\{\zero\}$ oraz $K$.
    \end{block}
    \pause 
    \begin{proof}
        Niech $\II \triangleleft K$. Jeżeli $\II = \{\zero\}$ to twierdzenie zachodzi. W innym przypadku istnieje $a \in K$ takie, że $a \neq \zero, a \in \II$. Z definicji dla dowolnego $r \in K$ mamy $r \cdot a \in K$. W szczególności dla $r = a^{-1}$ z czego wynika $a \cdot a^{-1} = 1 \in \II$ zatem $\II = K$.
    \end{proof}
    \pause 
    \begin{block}{Twierdzenie}
        Jeżeli jedynymi ideałami pierścienia $R$ są $\{\zero\}$ oraz całe $R$, to $R$ jest ciałem. 
    \end{block}
\end{frame}

\begin{frame}{Przecięcia Ideałów}
    \begin{block}{Twierdzenie}
        Niech będzie dany pierścień $R$ oraz niepusty podzbiór $T$. 
        Jeśli $\II_t \triangleleft R$ dla każdego $t \in T$ to $\bigcap_{t \in T} \II_t \triangleleft R$
        (przecięcie ideałów jest ideałem).
    \end{block}
\end{frame}

\begin{frame}{Ideały generowane}
    Rozważmy pierścień $R$ i element $a \in R$. Zdefiniujmy $\II = \{ r \cdot a \mid r \in R\}$. Zauważmy, że $\II$ jest ideałem $R$. Dodatkowo, zauważmy, że jest to najmnijeszy ideał zawierający $a$, to znaczy
    $$\{r \cdot a \mid r \in R\} = \bigcap \{\JJ \triangleleft R \mid a \in \JJ\}.$$
    \pause 
    Możemy to uogólnić:
    \begin{block}{Definicja}
        Niech $R$ będzie pierścieniem, natomiast $A = \{a_1, \ldots, a_s\} \subseteq R$ skończonym jego podzbiorem. 
        \alert{Ideałem generowanym przez A} nazywamy zbiór 
        $$\langle A \rangle = \langle a_1, \ldots , a_s \rangle = \left\{ \sum r_i \cdot a_i \mid r_i \in R, a_i \in A \right\}$$
    \end{block}
\end{frame}

\begin{frame}{Ideały generowane}
    \begin{block}{Twierdzenie}
        Ideał generowany przez zbiór jest ideałem.
    \end{block}
    \begin{block}{Twierdzenie}
        Niech $R$ będzie pierścieniem oraz $A = \{a_1, \ldots , a_s\} \subseteq R$.
        Zachodzi
        $$\langle A \rangle = \bigcap\{\II \triangleleft R \mid A \subseteq \II \}.$$
    \end{block}    
    \pause 
    \begin{alertblock}{Konwencja}
        Jeżeli $\II = \langle A \rangle = \langle a_1 , \ldots , a_s \rangle  \triangleleft R$, to $A$ nazywamy \textit{zbiorem generatorów} $\II$, elementy $a_1, \ldots , a_s$ nazywamy \textit{generatorami} $\II$. Jeżeli da się zapisać ideał jako generowany przez skończony zbiór, to mówimy, że ideał jest \textit{skończenie generowny}.
    \end{alertblock}
\end{frame}

\begin{frame}{Ideały generowane}
    \begin{alertblock}{Uwaga}
        Jeżeli podzbiór $A$ pierścienia $R$ jest nieskończony, wszystkie definicje i twierdzenia wciąż zachodzą, tylko dokładamy do definicji jeden warunek
        $$ \langle A \rangle = \left\{ \sum r_i \cdot a_i \mid r_i \in R, a_i \in A \right\}$$ gdzie \textit{prawie wszystkie} $r_i$ są równe 0.    
    \end{alertblock}
    \pause 
    \begin{alertblock}{Uwaga}
        Dowodzi się, że każdy ideał w $\mathbb{Z}$ oraz $K[x_1, \ldots , x_n]$, gdzie $K$ jest ciałem jest skończenie generowany.
    \end{alertblock}
\end{frame}

\begin{frame}{Suma Algebraiczna Ideałów}
    \begin{block}{Definicja}
        Niech $R$ bedzie pierścieniem oraz $\II, \JJ \triangleleft R$. \alert{Sumę ideałów $\II, \JJ$} definiujemy jako
        $$\II + \JJ = \{ a+b \mid a \in \II, b \in \JJ\}.$$
    \end{block}
    \pause 
    \begin{block}{Twierdzenie}
        Jeżeli $\II , \JJ \triangleleft R$, gdzie $R$ jest pierścieniem, to $\II + \JJ \triangleleft R$ (suma ideałów jest ideałem). 
    \end{block}
    \pause
    \begin{block}{Twierdzenie}
        Jeżeli $\II = \langle a_1, \ldots, a_s \rangle$ oraz $\JJ = \langle b_1, \ldots, b_r \rangle$ to 
        $$ \II + \JJ = \langle a_1, \ldots a_s, b_1, \ldots, b_r \rangle.$$
    \end{block}
\end{frame}

\begin{frame}{Względna pierwszość Ideałów}
    \begin{block}{Definicja}
        Mówimy, że dwa ideały $\II, \JJ$ pierścienia $R$ są \alert{względnie pierwsze} jeśli zachodzi
        $$\II + \JJ = R$$
    \end{block}
\end{frame}

\begin{frame}{Iloczyn algebraiczny Ideałów}
    \begin{block}{Definicja}
        Niech $R$ będzie pierścieniem oraz $\II, \JJ$ ideałami w tym pierścieniu. \alert{Iloczyn algebraiczny ideałów $\II, \JJ$} definiujemy jako
        $$ \II \JJ = \left\{ \sum a_i \cdot b_i \mid a _i\in \II, b_i \in \JJ \right\}.$$
        gdzie prawie wszystkie $a_i \cdot b_i$ są zerami (są to skończone sumy).
    \end{block}
    \pause
    \begin{block}{Twierdzenie}
        Iloczyn algebraiczny ideałów jest ideałem. 
    \end{block}
    \begin{alertblock}{Obserwacja}
        Jeśli $\II_1, \ldots \II_n$ to ideały pierścienia $R$, to zauważmy, że 
        $$ \II_1 \cdots \II_n =  \langle a_{i1} \cdots a_{in} \mid a_{ij} \in I_j \rangle$$
    \end{alertblock}
\end{frame}

\begin{frame}{Iloczyn algebraiczny ideałów}
    \begin{block}{Twierdzenie}
        Jeśli $\II_1, \ldots \II_n$ to ideały pierścienia $R$, to zachodzi
        $$\II_1 \cdots \II_n \subseteq \II_1 \cap \ldots \cap \II_n $$
    \end{block}
    \pause
    \begin{block}{Twierdzenie}
        Niech $R$ będzie pierścieniem oraz $\II_1 , 
        \ldots \II_n \triangleleft R$ będą ideałami (parami) względnie pierwszymi w $R$. Zachodzi
        $$\II_1 \cap \ldots \cap \II_n = \II_1 \cdot \ldots \cdot \II_n$$
    \end{block}
\end{frame}

\begin{frame}{Ideały Pierwsze}
    \begin{block}{Definicja}
        Ideał $\II$ pierścienia $R$ nazywamy \alert{ideałem pierwszym}, jeżeli zachodzi 
        $$\forall a,b \in R \quad ab \in \II \Rightarrow a \in \II \lor b \in \II .$$
    \end{block}
\end{frame}

\section{Pierścień Ilorazowy}
\begin{frame}{Warstwy Ideału}
    Rozważmy pierścień $R$ oraz jego ideał $\II$.
    \begin{alertblock}{Obserwacja}
        Ponieważ $(\II, +)$ jest podgrupą (normalną) addytywną pierścienia, konstrukcja warstw oraz grupy ilorazowej ze względu na dodawanie jest w pełni poprawna.
    \end{alertblock}
    \begin{alertblock}{Konwencja}
        Zwyczajowo warstwę elementu $a \in R$ względem ideału $\II$ oznaczamy jako $a + \II.$ 
    \end{alertblock}
\end{frame}

\begin{frame}{Mnożenie Warstw}
    \begin{block}{Twierdzenie}
        Jeśli $\II$ jest ideałem pierścienia $R$ to wzór
        $$(a + \II) \odot (b + \II) = ab + \II$$
        określa działanie w zbiorze warstw względem $\II$ nazywane mnożeniem warstw. Zbiór warstw z dodawaniem warstw i mnożeniem warstw tworzy pierścień.
    \end{block}
\end{frame}

\begin{frame}{Pierścień Ilorazowy}
    \begin{block}{Definicja}
        Jeżeli $\II$ jest ideałem pierścienia $R$ to pierścień warstw z dodawaniem warstw i mnożeniem warstw nazywamy \alert{Pierścieniem Ilorazowym} $R$ względem $\II$ i oznaczamy przez $R/\II$.
    \end{block}
    \pause 
    \begin{alertblock}{Konwencja}
        Warstwe $a + \II$ nazywamy klasą reszt $a$ albo klasą modulo $a$ albo klasą abstrakcji $a$. \\
        $R/\II$ czasem nazywane jest też pierścieniem reszt modulo $\II$. \\
        Działania w pierścieniu ilorazowym $R/\II$ oznacza się zwykle tak samo jak działania w pierścieniu $R$.
    \end{alertblock}
\end{frame}

\section{Jądro Homomorfizmu}
\begin{frame}{Homomorfizmy pierścieni}
    \begin{block}{Twierdzenie}
        Niech $R, R'$ będą pierścieniami. Jądro homomorfizmu $\phi : R \rightarrow R'$ jest ideałem pierścienia $R$.
    \end{block}
    \pause 
    \begin{block}{Twierdzenie}
        Niech $R, R'$ będą pierścieniami, $\phi : R \rightarrow R'$ homomorfizmem tych pierścieni oraz $\JJ \triangleleft R'$. Zachodzi $\phi^{-1}(\JJ) \triangleleft R$ (przeciwobraz ideału jest ideałem).
    \end{block}
    \pause
    \begin{alertblock}{Uwaga}
        Obrazem homomorficznym ideału \alert{nie zawsze} jest ideał.
    \end{alertblock}
\end{frame}

\begin{frame}{Charakterystyka jądra}
    \begin{block}{Twierdzenie}
        Jeśli $\II$ jest ideałem pierścienia $R$ to odwzorowanie $\nu : R \rightarrow R/\II$ zadane wzorem $\nu(a) = a + \II$ jest homomorfizmem pierścieni $R, R/\II$ oraz $\ker \nu = \II$. 
    \end{block}
    \pause
    \begin{alertblock}{Konwencja}
        Homomorfizm $\nu : R \rightarrow R/\II$ nazywamy homomorfizmem kanonicznym.
    \end{alertblock}
    \pause 
    \begin{alertblock}{Wniosek}
        Ideały danego pierścienia i tylko one są jądrami homomorfizmów pierścieni.
    \end{alertblock}
\end{frame}

\begin{frame}{Obraz homomorficzny pierścienia}
    \begin{block}{Twierdzenie}
        Jeśli $\phi$ jest homomorfizmem pierścienia $R$ na pierścień $R'$, to istnieje izomorfizm $\psi : R' \rightarrow R/\ker\phi$,
        dla którego przemienny jest diagram 
        $$
            \begin{tikzcd}[ampersand replacement=\&, column sep=small]
                R \arrow[r, "\phi"]  \arrow[rd, "\nu"'] \& R' \arrow[d, "\psi"] \\
                \& R / \ker \phi
            \end{tikzcd}
        $$   
        gdzie $\nu$ to homomorfizm kanoniczny.
    \end{block}
    \pause
    \begin{block}{Twierdzenie}
        Jedynymi homomorfizmami ciała na ciało są izomorfizmy. 
    \end{block}
\end{frame}

\section{Kongruencje Pierścieni}
\begin{frame}{Kongruencje}
    \begin{block}{Definicja}
        Relacja równoważności $\sim$ określona na pierścieniu $R$ nazywa się \alert{kongruencją}, jeżeli 
        $$[(a \sim b) \land (c \sim d)] \Rightarrow [(a+c \sim b+d) \land (ac \sim bd)].$$
    \end{block}
    Kongruencje, to relacje równoważności, ktore można dodawać i mnożyc stronami.
\end{frame}

\begin{frame}{Przystawanie modulo Ideał}
    \begin{block}{Twierdzenie}
        Niech $\II$ będzie ideałem pierścienia $R$. Relacja przystawania elementów modulo $\II$
        $$ a \equiv b \ (\II) \Leftrightarrow a - b \in \II$$ 
        jest kongruencją w pierścieniu $R$. 
    \end{block}
    \pause
    \begin{alertblock}{Obserwacja}
        Kongruencje pierścieni można odejmować stronami. 
    \end{alertblock}
\end{frame}

\begin{frame}{Charakterystyka Kongruencji}
    \begin{block}{Twierdzenie}
        Jeśli relacja równoważności $\sim$ określona na pierścieniu $R$ jest kongruencją, to klasa abstrakcji $\II$ zbioru ilorazowego $R/\sim$ która zawiera $\zero$ pierścienia $R$ jest ideałem pierścienia $R$ oraz $R/\! \! \sim  \ = \  R/\II$.
    \end{block}
\end{frame}

\begin{frame}{Źródła}
    \printbibliography
\end{frame}

\begin{frame}
    \centering 
    \LARGE Pytania, wątpliwości, uwagi ? 
\end{frame}
\end{document}